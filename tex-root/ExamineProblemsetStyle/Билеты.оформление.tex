\documentclass[a4paper, 11pt]{article}
\usepackage[T2A]{fontenc}
\usepackage[utf8]{inputenc}
\usepackage[russian]{babel}
\usepackage{latexsym,enumerate}
\usepackage{amssymb,amsfonts,amsthm}

\textwidth 16cm
\textheight 27cm
\voffset -1.5in
\hoffset -0.8in

\newcommand{\HRule}{\rule{\linewidth}{0.1mm}}
\newcommand{\mrule}[1]{${}_\textrm{\rule{#1}{0.1mm}}$}
\newcommand{\Header}[1]{
  \section*{\centerline{Вопросы к экзамену}\\ \centerline{по курсу <<Дискретная математика>>}}

  {\small
    \rightline{``Утверждаю''}
    \rightline{Заведующий кафедой}
    \rightline{Алгебры и дискретной математики}
    \rightline{\mrule{3.1cm} /Волков М.В./}
    \rightline{2 апреля 2009 года}
  }

  \subsection*{\centerline{Билет #1}}
}


\pagestyle{empty}
\begin{document}

\HRule

\Header{1}
\begin{enumerate}
  \item Двусвязность, компоненты двусвязности. Точки сочленения, мосты.
  \item На танцплощадке собрались $n$  юношей и $n$  девушек. Сколькими способами они могут разбиться на пары для участия в очередном танце? 
\end{enumerate}

\HRule

\Header{2}
\begin{enumerate}
  \item Деревья. Теорема о деревьях. Цикломатическое число. Каркас.
  \item Сколькими способами можно выбрать из колоды в 52 карты 6 карт так, чтобы среди них было не более 3 шестерок?
\end{enumerate}

\HRule
	
\Header{3}
\begin{enumerate}
  \item Раскраски. Оценки хроматического числа.
  \item Найти число всех таких слов длины $mn$ в $n$-буквенном алфавите, в которых каждая буква алфавита встречается $m$ раз.
\end{enumerate}

\HRule

\pagebreak

\HRule

\Header{4}
\begin{enumerate}
  \item Планарность. Укладка на сфере. Теорема Понтрягина-Куратовского (доказательство необходимости).
  \item Найти число векторов $\vec{x}=(x_1,\ldots,x_n)$, координаты которых удовлетворяют условиям
  \begin{enumerate}
    \item[(а)] $x_i\in\{0,1,\ldots,k-1\} (i=1,\ldots,n)$;
    \item[(б)] $x_i\in\{0,1\} (i=1,\ldots,n)$ и $x_1+\ldots+x_n=r$.
  \end{enumerate}
\end{enumerate}

\HRule

\Header{5}
\begin{enumerate}
  \item Раскраска плоского графа. Теорема о пяти красках.
  \item Найти число всех таких слов длины $mn$ в $n$-буквенном алфавите, в которых каждая буква алфавита встречается $m$ раз.
\end{enumerate}

\HRule

\Header{6}
\begin{enumerate}
  \item Орграф. Сильная связность. Эйлеров и гамильтонов орцикл.
  \item Сколько существует 6-значных чисел, у которых каждая последующая цифра меньше предыдущей?
\end{enumerate}

\HRule

\end{document}
