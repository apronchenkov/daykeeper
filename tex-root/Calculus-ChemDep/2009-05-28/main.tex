% 2009-01-14
% 2009-01-11
% 2009-01-10

\documentclass[10pt,a4paper]{scrartcl}

%%
% Пакет позволяющий определить, что используется: latex или pdflatex?
%%
\usepackage{ifpdf}

%%
% Пакеты AMS*
%%
\usepackage{amsfonts, amsmath, amsthm}

%%
% Определяем, используется ли XeTex?
%%
\ifx\XeTeXversion\undefined
  %%
  % Набор пакетов для работы с графическими файлами
  %%
  \ifpdf
    \usepackage[pdftex]{graphicx}
    \usepackage{cmap}
  \else
    \usepackage{graphicx}
  \fi

  \usepackage[utf8x]{inputenc} 

\else
  %%
  % Набор пакетов для поддержки русского языка в XeLaTeX
  %%
  \usepackage[cm-default]{fontspec}
  \usepackage{xunicode}
  \usepackage{xecyr}

  % Setting default fonts
  \usepackage{unicode-math}
  \setmainfont[Mapping=tex-text]{Cambria}
  \setmathfont{Cambria Math}
  
  %\setmainfont{Times New Roman} 
  %\setmainfont{Georgia}
  %\setmainfont{Lucida Console}
\fi

%%
% Дополнительные настройки
%%
%\usepackage{geometry} % пакет для установки полей
%\geometry{top=.7cm} % отступ сверху
%\geometry{bottom=1cm} % отступ снизу
%\geometry{left=2cm} % отступ справа
%\geometry{right=2cm} % отступ слева


\usepackage[english,russian]{babel} 
\usepackage{indentfirst}


\newcommand{\exercize}[1]{\textbf{#1.}}
\newcommand{\abs}[1]{\lvert{}#1{}\rvert}
\newcommand{\HRule}{\rule{\linewidth}{0.1mm}}

%%
% Начало документа
%%
\begin{document}

\HRule

\section{Построение графиков функций по характерным точкам}

Для построения графика функции $y = f(x)$ нужно выполнить следующие шаги.
\begin{enumerate}
  \item {Определить область существования этой функции и исследовать поведение функции в граничных точках последней.}
  \item {Выяснить симметрию графика и периодичность.}
  \item {Найти точки разрыва функции и промежутки непрерывности.}
  \item {Определить нули функции и области постоянства знака.}
  \item {Найти точки экстремума и выяснить промежутки возрастания и убывания функции.}
  \item {Определить точки перегиба и установить промежутки вогнутости определённого знака графика функции.}
  \item {Найти асимптоты в случае существования их.}
  \item {Указать те или иные особенности графика.}
\end{enumerate}

В частных случая общая схема упрощается.

\HRule

\section{Построить графики следующих функций}

\begin{eqnarray}
  y & = & 3x - x^3. \\
  y & = & (x + 1)(x - 2)^3. \\
  y & = & \frac{2 - x^2}{1 + x^4} \text{(точки перегиба искать приближённо)}. \\
  y & = & (x - 3) \sqrt{x}. \\
  y & = & \sin{x} + \cos^3{x}. \\
  y & = & (7 + 2 \cos{x})\sin{x}. \\
  y & = & \sin{x} + \frac{1}{3}\sin{3 x}. \\
  y & = & \cos{x} - \frac{1}{2}\cos{2 x}. \\
  y & = & \sin^4{x} + \cos^4{x}. \\
  y & = & \sin{x} \sin{3 x}. \\
  y & = & \frac{e^x}{1 + x}.
\end{eqnarray}


\section{Найти производные следующих функций}

\begin{eqnarray}
  y & = & x + \sqrt{x} + \sqrt[3]{x}. \\
  y & = & \frac{1}{x} + \frac{1}{\sqrt{x}} + \frac{1}{\sqrt[3]{x}}. \\
  y & = & \arctg{\sqrt{\sqrt{x} - 1}}. \\
  y & = & \sqrt{x + \sqrt{x + \sqrt{x}}}. \\
  y & = & \sqrt[3]{1 + \sqrt[3]{1 + \sqrt[3]{x}}}. \\
  y & = & \cos{2 x} - 2 \sin{x}. \\
  y & = & \frac{\sin^2{x}}{\sin{x^3}}. \\
  y & = & \frac{\cos{x}}{2 \sin^2{x}}. \\
  y & = & \frac{1}{\cos^n{x}}. \\
  y & = & \tg{\frac{x}{2}} - \ctg{\frac{x}{2}}. \\
  y & = & e^x (1 + \ctg{\frac{x}{2}}). \\
  y & = & e^x + e^{e^x} + e^{e e^x}.
\end{eqnarray}

\section{Найти следующие определённые интегралы}

\begin{eqnarray}
  & \int\limits_{0}^{2\pi} {\frac{dx}{2 + \cos{x}}}. & \\
  & \int\limits_{0}^{100\pi} {\sqrt{1 - \cos{2 x} dx}}. & \\
  & \int\limits_{0}^{2\pi} {x^2 \cos{x} dx}. & \\
  & \int\limits_{0}^{2} {|1 - x| dx}. & \\
  & \int\limits_{0}^{1} {x |x - \frac{1}{3}| dx}. & \\
  & \int\limits_{1/e}^{e} {|\ln{x}| dx}. & \\
  & \int\limits_{0}{\ln{2}} {\sqrt{e^x - 1} dx}. \\
  & \int\limits_{0}^{\pi} {\frac{x \sin{x}}{1 + \cos^2{x}} dx}. & \\
  & \int\limits_{0}^{\sqrt{3}} {x \arctg{x} dx}. & \\
  & \int\limits_{-1}^{1} {\frac{x dx}{\sqrt{5 - 4x}}}. & \\
  & \int\limits_{1}^{2} {\frac{dx}{x (x^2 + 1)}}. & \\
\end{eqnarray}

\end{document}
