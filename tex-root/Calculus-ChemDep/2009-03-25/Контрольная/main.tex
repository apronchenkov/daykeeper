% 2009-01-14
% 2009-01-11
% 2009-01-10

\documentclass[a4paper,10pt]{article}

%%
% Пакет позволяющий определить, что используется: latex или pdflatex?
%%
\usepackage{ifpdf}

%%
% Пакеты AMS*
%%
\usepackage{amsfonts, amsmath, amsthm}

%%
% Определяем, используется ли XeTex?
%%
\ifx\XeTeXversion\undefined
  %%
  % Набор пакетов для работы с графическими файлами
  %%
  \ifpdf
    \usepackage[pdftex]{graphicx}
    \usepackage{cmap}
  \else
    \usepackage{graphicx}
  \fi

  \usepackage[utf8x]{inputenc} 

\else
  %%
  % Набор пакетов для поддержки русского языка в XeLaTeX
  %%
  \usepackage[cm-default]{fontspec}
  \usepackage{xunicode}
  \usepackage{xecyr}

  % Setting default fonts
  \usepackage{unicode-math}
  \setmainfont[Mapping=tex-text]{Cambria}
  \setmathfont{Cambria Math}
  
  %\setmainfont{Times New Roman} 
  %\setmainfont{Georgia}
  %\setmainfont{Lucida Console}
\fi

%%
% Дополнительные настройки
%%
\usepackage[english,russian]{babel} 
\usepackage{indentfirst}


\newcommand{\exercize}[1]{\textbf{#1.}}
\newcommand{\abs}[1]{\lvert{}#1{}\rvert}
\newcommand{\HRule}{\rule{\linewidth}{0.1mm}}

%%
% Начало документа
%%
\begin{document}

\HRule
\section{Вариант}
% \exercize{293}
1. Найти предел
\[ \lim_{x \rightarrow 0}{\frac{\sqrt{1 + x^2} - 1}{x}}. \]

% \exercize{498}
2. Продифференцировать функцию:
\[ (x - a)(x - b)(x - c)(x - d). \]

% \exercize{796}
3. Найти производную функции $y,$ заданной неявно: $ y^3 - 3y + 2 a x = 0. $

% \exercize{204}
4. Доказать, что функция: 
\[ y = \frac{x^2}{1 + x^4} \]
--- ограничена на всей числовой оси.

\HRule

\section{Вариант}
% \exercize{294}
1. Найти предел
\[ \lim_{x \rightarrow 0}{\frac{\sqrt{1 + x} - 1}{x^2}}. \]

% \exercize{553}
2. Продифференцировать функцию: 
\[ x \sin{x} \arctg{x}. \]

% \exercize{798}
3. Найти производную функции $y,$ заданной неявно: $ x^4 + y^4 = x^2 y^2. $

% \exercize{1107} 
4. Показать, что точка $x = 0$ есть точка минимума функции
\[ y = 3 x^4 - 4 x^3 + 12 x^2 + 1. \]

\HRule


\section{Вариант} 
% \exercize{302} 
1. Найти предел
\[ \lim_{x \rightarrow 1}{\frac{\sqrt[n]{x} - 1}{\sqrt[m]{x} - 1}}, \quad \textrm{($n$ и $m$ --- целые числа).} \]

% \exercize{612}
2. Продифференцировать функцию:
\[ (x^2 - 2x + 3) e^x. \]

% \exercize{805}
3. Найти производную функции $y,$ заданной неявно: $ y = \cos{(x + y)}. $

% \exercize{1317}
4. Найти точки перегиба линии
\[ x = e^t, y = \sin{t}. \]

\HRule
\newpage
\HRule

\section{Вариант} 

% \exercize{361}
1. Найти предел
\[ \lim_{x \rightarrow \pm \infty}{\left(1 + \frac{1}{x}\right)^{x^2}}. \]

% \exercize{692}
2. Продифференцировать функцию:
\[ \arcsin{(n \sin{x})}. \]

% \exercize{1062}
3. Найти вторую производную функции $y,$ заданной неявно: $ e^{x + y} = x y. $

% \exercize{1316}
4. Найти точки перегиба линии 
\[ x = t^2, \quad y = 3t + t^3. \]

\HRule

\section{Вариант} 

% \exercize{1346}
1. Найти предел
\[ \lim_{x \rightarrow + \infty}{x^n e^{-x}}. \]

% \exercize{752}
2. Продифференцировать функцию:
\[ \ln{(x \sin{x} \sqrt{1 - x^2})}. \]

% \exercize{944}
3. Найти производную $y$ по $x:$ 
\[ x = e^t \sin{t}, \quad y = e^t \cos{t}. \]

% \exercize{1307}
4. При каких значениях параметра $a$ график функции
\[ y = e^x + a x^3 \]
имеет точки перегиба?

\HRule

\end{document}
