% 2009-01-14
% 2009-01-11
% 2009-01-10

\documentclass[a4paper,10pt]{article}

%%
% Пакет позволяющий определить, что используется: latex или pdflatex?
%%
\usepackage{ifpdf}

%%
% Пакеты AMS*
%%
\usepackage{amsfonts, amsmath, amsthm}

%%
% Определяем, используется ли XeTex?
%%
\ifx\XeTeXversion\undefined
  %%
  % Набор пакетов для работы с графическими файлами
  %%
  \ifpdf
    \usepackage[pdftex]{graphicx}
    \usepackage{cmap}
  \else
    \usepackage{graphicx}
  \fi

  \usepackage[utf8x]{inputenc} 

\else
  %%
  % Набор пакетов для поддержки русского языка в XeLaTeX
  %%
  \usepackage[cm-default]{fontspec}
  \usepackage{xunicode}
  \usepackage{xecyr}

  % Setting default fonts
  \usepackage{unicode-math}
  \setmainfont[Mapping=tex-text]{Cambria}
  \setmathfont{Cambria Math}
  
  %\setmainfont{Times New Roman} 
  %\setmainfont{Georgia}
  %\setmainfont{Lucida Console}
\fi

%%
% Дополнительные настройки
%%
\usepackage[english,russian]{babel} 
\usepackage{indentfirst}


\newcommand{\exercize}[1]{\textbf{#1.}}
\newcommand{\abs}[1]{\lvert{}#1{}\rvert}

%%
% Начало документа
%%
\begin{document}

\section{Домашнее задание}

\exercize{1069} \[ x = a t^2,\ y = b t^3;\ \frac{d^2 x}{{dy}^2} = ? \]

\exercize{1070} \[ x = a \cos{t},\ y = a \sin{t};\ \frac{d^2 x}{{dy}^2} = ? \]

В задачах 1324 -- 1364 найти пределы.

\exercize{1325} \[ \lim_{x \rightarrow 0} {\frac{\ln{\cos{x}}}{x}}. \]

\exercize{1334} \[ \lim_{x \rightarrow 0} {\frac{e^{x^2} - 1}{\cos{x} - 1}}. \]

\exercize{1344} \[ \lim_{x \rightarrow 0} {\frac{\ln{x}}{\ln{\sin{X}}}}. \]

\exercize{1358} \[ \lim_{x \rightarrow 0} {x^{\sin{x}}}. \]

\exercize{1363} \[ \lim_{x \rightarrow 0} {\left( 1 + \frac{1}{x^2} \right)^x}. \]


\section{Работа на занятии}

\exercize{1108} Исходя непосредственно из определения возрастающей и убывающей функции и точек максимума и минимума, показать, что функция 
\[ y = x^3 - 3 x + 2 \]
возрастает в точке $x_1 = 2,$ убывает в точке $x_2 = 0,$ достигает масимума в точке $x_3 = -1$ и минимума в точке $x_4 = 1$.

\exercize{1109} Так же, как в задаче 1108, показать, что функция 
\[ y = \cos{2 x} \]
возрастает в точке $x_1 = 3 \pi / 4,$ убывает в точке $x_2 = \pi / 6,$ достигает максимума в точке $x_3 = 0$ и минимума в точке $x_4 = \pi / 2.$

В задачах 1165 -- 1184 найти экстремумы функций.

\exercize{1176} \[ y = x - \ln{(1 + x^2)}. \]

\exercize{1179} \[ y = \frac{1}{2}(x^2 + 1) \arctg{x} - \frac{\pi}{8}x^2 - \frac{x - 1}{2}. \]


В задачах 1185 -- 1197 найти наибольшие и наименьшие значения данных функций на указанных отрезках и в указанных интервалах.

\exercize{1197} \[ y = \arctg{\frac{1 - x}{1 + x}},\ (0 \le x \le 1). \]

В задачах 1198 -- 1207 доказать справедливость неравенств.

\exercize{1199} \[ e^x > 1 + x,\ (x \not= 0). \]

\exercize{1200} \[ x > \ln{(1 + x)},\ (x > 0). \]

\exercize{1203} \[ 1 + x \ln{(x + \sqrt{1 + x^2})} \ge \sqrt{1 + x^2}. \]

В задачах 1398 -- 1464 провести полное исследование данных функций и начертить их графики.

\exercize{1398} \[ y = \frac{x}{1 + x^2}. \]

\exercize{1402} \[ y = \frac{x^2}{x^2 - 1}. \]

\exercize{1412} \[ y = \frac{(x - 1)^2}{(x + 1)^3}. \]

\exercize{1416} \[ y = \frac{x}{e^x}. \]

\exercize{1417} \[ y = x^2 e^{-x}. \]

\exercize{1425} \[ y = x + \frac{\ln{x}}{x}. \]

\exercize{1252, 1253}

\end{document}
