%
% Данный реферативный текст был подготовлен в рамках курса философия науки (кандидатский минимум).
%

% 2009-01-14
% 2009-01-11
% 2009-01-10

\documentclass[a4paper,14pt]{scrartcl}

%%
% Пакет позволяющий определить, что используется: latex или pdflatex?
%%
\usepackage{ifpdf}

%%
% Пакеты AMS*
%%
\usepackage{amsfonts, amsmath, amsthm}

%%
% Набор пакетов для работы с графическими файлами
%%
\ifpdf
  \usepackage[pdftex]{graphicx}
  \usepackage{cmap}
\else
  \usepackage{graphicx}
\fi

\usepackage[utf8x]{inputenc} 

%%
% Дополнительные настройки
%%
\usepackage[english,russian]{babel} 
\usepackage{indentfirst}

\usepackage{setspace}
\onehalfspacing


\newcommand{\exercize}[1]{\textbf{#1.}}
\newcommand{\abs}[1]{\lvert{}#1{}\rvert}

%%
% Начало документа
%%
\begin{document}

\title{Тест Тьюринга: его методологическое и практическое значение}
\author{Александр Геннадьевич Пронченков}

% \maketitle

\tableofcontents

\thispagestyle{empty}

\pagebreak

\section{Введение}

Как и многие другие разделы компьютерных наук, теория искусственного интеллекта началась с работы Алана Тьюринга \cite{turing_1936}. В этой работе было дано определение {\it машины Тьюринга,} и автор показал, что эта машина способна моделировать шаг за шагом действия человека, решающего на бумаге некоторую задачу.

Появление машины Тьюринга позволило формализовать понятие {\it алгоритм.} Сейчас известны и другие формализации алгоритма, например, определения Алонзо Черча \cite{church_1932}, Стивена Клини \cite{kleene_1936} или Андрея Маркова \cite{markov_1984}. Все они отражают суть неформального понятия {\it решение задачи.} И принято считать, что эти определения (а также многие другие) эквивалентны друг другу. Тем не менее, именно машина Тьюринга стала моделью {\it универсального вычислителя}. До ее изобретения, для каждой решаемой задачи изобретался собственный вычислитель.

Естественно поставить вопрос: раз машина Тьюринга универсальна, то может ли она мыслить? Этот вопрос был рассмотрен Тьюрингом в работе \cite{turing_1950}.

\section{Тест Тьюринга}

Работа \cite{turing_1950} начинается со слов: <<Я собираюсь рассмотреть вопрос: могут ли машины мыслить?>> И тут же автор же замечает, что, прежде чем удастся ответить на этот вопрос, необходимо определиться с понятиями: {\it машина} и {\it мыслить.} Но вместо того, чтобы пытаться ввести такие определения, Тьюринг предлагает изменить формулировку вопроса.

Новый вопрос формулируется Тьюрингом через {\it игру в имитацию.} В этой игре участвуют три человека: мужчина, женщина и {\it кто-нибудь задающий вопросы}, которым может быть лицо любого пола. Задающий вопросы отделен от двух других участников игры стенами комнаты, в которой он находится. Цель игры для задающего вопросы состоит в том, чтобы определить, кто из двух других участников игры является мужчиной, а кто — женщиной. Цель для игрока-мужчины состоит в том, чтобы привести задающего вопросы к неверному заключению. Цель игры {\bf второго игрока} — {\bf женщины} — состоит в том, чтобы помочь задающему вопросы.

Далее Тьюринг предлагает вместо старого: <<Может ли машина мыслить?>> — новый вопрос: <<Что будет, если машина возьмёт на себя роль игрока-мужчины? Будет ли в этом случае задающий вопросы ошибаться столь же часто, как и в игре, где участниками являются только люди?>>

Отметим вслед за \cite{sayagin_2000} следующую неоднозначность. В оригинальной статье присутствовала ещё одна формулировка вопроса \cite[стр. 442]{turing_1950}: <<Можно ли заставить машину удовлетворительно исполнять роль мужчины в игре, где роль {\bf второго игрока} играет {\bf человек?}>> В оставшейся части своей статьи Тьюринг игнорировал вопрос {\it рода.} И большинство последовавших работ также игнорировали этот вопрос, предполагая, что цель задающего вопросы — определить, кто из игроков является машиной.

\section{Дискуссии об искусственном интеллекте}

Статья Тьюринга стала отправной точкой в дискуссии о существовании искусственного интеллекта. И, что естественно, направление этим дискуссиям было задано формулировкой теста Тьюринга, которая, фактически, определяет мышление как способность имитировать человеческие действия. Подобное представление об интеллекте стало характерным для ранних исследований в области искусственного интеллекта.

Когда Тьюринг писал статью <<Могут ли машины мыслить>>, он понимал, что его идеи  будут подвергнуты критике, и в своей статье сразу же дал ответ на часть потенциальных возражений. Мы рассмотрим некоторые из них.

% \subsection{Возражение со <<страусиной>> точки зрения}

% Суть этого возражения заключается в следующем: {\it думающие машины} могут привести к опасным, ужасающим последствиям, а потому следует надеяться, что такие машины не могут существовать.

% Тьюринг верил, что данный аргумент даже не требует опровержения. С долей сарказма он заключает, что утешение в данном случае было бы более уместным (не поискать ли его, например, в учении о переселении душ?)

% Сейчас подобная точка зрения распространена менее, чем во времена Тьюринга.

\subsection{Математическое возражение}

Данное возражение основано на теоретической неразрешимости некоторых математических проблем. Наиболее известный пример — теорема Гёделя. Она утверждает, что в любой достаточно мощной непротиворечивой логической системе можно формулировать такие утверждения, которые внутри этой системы нельзя ни доказать, ни опровергнуть. Аналогичные результаты есть и в теории алгоритмов. Таким образом, возможности любой машины или логической системы заранее ограничены. 

Тьюринг, однако, отметил отсутствие каких-либо доказательств того, что для человека подобные ограничения не применимы.

Более того, он считает, что основой это возражение служит предубеждение, что мыслящая машина не может ошибаться. Последнее, очевидно, не является необходимым требованием для мышления.

\subsection{Возражение с точки зрения сознания}

Возможно, одно из самых важных возражений — это возражение с точки зрения сознания.

Многие люди уверены, что для того, чтобы машина могла мыслить, она должна обладать сознанием (т.е. получала удовольствие от успеха, огорчалась при неудачах, осознавала свою незавершённость и т. д.) Фактически, эта точка зрения является {\it солипсистской:} единственный путь {\it действительно} узнать, обладает или нет машина сознанием, — быть этой машиной. Однако, согласно этой же точке зрения, единственный путь узнать, мыслит ли человек, — оказаться этим человеком. В английском языке такая проблема имеет название {\it other minds problem.} \footnote{(дословно) проблема других сознаний.}

Подобные возражения несколько раз возникали в дискуссиях о Тесте Тьюринга. Сам же Тьюринг отмечал, что при общении между людьми <<вместо того чтобы постоянно спорить по этому вопросу, обычно принимают вежливое соглашение о том, что мыслят все.>>

Тем не менее, Тьюринг не считал это возражение беспочвенным. Он просто верил, что для ответа на вопрос о мышлении, нет необходимости решать эту загадку \cite[стр. 447]{turing_1950}.


\subsection{Возражение, основанное на непрерывности действия нервной системы}

В своей статье Тьюринг написал: <<Нет сомнения в том, что нервная система не является машиной с дискретными состояниями\ldots Исходя из этого, можно было бы как будто предполагать, что нельзя имитировать поведение нервной системы с помощью машины с дискретными состояниями.>> И затем сам себе возразил: <<То, что машина с дискретными состояниями должна отличаться от машины непрерывного действия, это, конечно, справедливо. Однако если мы будем придерживаться условий ``игры в имитацию'', то задающий вопросы не сможет использовать это различие.>>

Тьюринг считал, что непрерывность состояний нивелируется, и мозг вполне может быть приближен дискретным устройством так же, как возможно вычисление непрерывной функции с заданной погрешностью.


\subsection{Возражение леди Лавлейс}

Наиболее подробные сведения об {\it Аналитической машине} Бэббиджа, прообразе современных вычислителей, беруться из воспоминаний леди Лавлейс. В своих работах она высказала следующую мысль: <<Аналитическая машина не претендует на то, чтобы создавать что-то действительно новое. Машина может выполнить все то, что мы умеем ей предписать.>>

В качестве контраргумента Тьюринг приводит следующее высказывание: <<Мнение о том, что машины не могут чем-либо удивить человека, основывается, как я полагаю, на одном заблуждении, которому в особенности подвержены математики и философы. Я имею в виду предположение, что коль скоро какой-то факт стал достоянием разума, тотчас же достоянием разума становятся все следствия из этого факта. Во многих случаях это предположение может быть весьма полезно, но слишком часто забывают, что оно ложно.>>

\subsection{Бихевиоризм и Нед Блок}

Возражение Неда Блока \cite{block_1981} основано на построении воображаемой машины специального вида. 

В любой беседе может быть задано лишь конечное число вопросов, и на каждый вопрос существует ответ, который удовлетворит жюри, принимающее Тест Тьюринга. Соответственно, мы можем запрограммировать машину так, чтобы она просто давала правильные ответы на любые вопросы, которые может задавать жюри. Поскольку вопросов конечное число, то потенциально такая машина может быть построена. Соответственно, Тест Тьюринга может быть пройден. Но совершенно очевидно, что такая машина не обладает мышлением ни в какой интерпретации этого термина. 

Несложно понять, что машина, созданная описанным образом, окажется слишком велика. Число вопросов действительно конечно, однако поскольку ответы на вопросы связаны между собой, количество вариантов становится астрономическим. Ответ на второй вопрос зависит от первого ответа. Ответ на третий вопрос зависит от первых двух. И т. д. Уже для получасовой беседы число вариантов может легко превысить число частиц в известной части вселенной. Поэтому такая машина невозможна, а значит, она не может выступать основой для логических построений.

Позволю себе отметить, что на мой взгляд, данное возражение вступает в спор с представлениями Платона о душе. Машина уже как будто знает всё заранее и лишь вспоминает по ходу беседы. Возможно, Платон не согласился бы с тем, что подобная машина неразумна.

\subsection{Китайская комната и Джон Сирл}

Следующий мысленный эксперимент, получивший название «пример Китайской комнаты», был предложен Джоном Сирлом в 1980 году \cite{searle_1980}.

Представим себе машину Тьюринга, которая проходит тест на некотором иностранном языке. Для определенности, на китайском. Программа машины Тьюринга может быть снята с машины и записана в некой книге правил. Далее Сирл представляет, что человек берет эту книгу, запирается в комнате и начинает общаться с окружающим миром через текстовые сообщения на китайском. Причём свои ответы человек получает путём преобразования поступающих ему текстов в соответствии с книгой правил. Фактически, он играет роль машины Тьюринга. Поскольку исходная машина прошла Тест Тьюринга, окружающие будут считать, что человек, использующий книгу правил, знает китайский язык. В это же самое время, человек, находящийся внутри комнаты, может воспринимать получаемые им тексты лишь как набор рисунков.

Из этого Сирл заключает, что прохождение теста Тьюринга не означает, что машина понимает информацию, с которой имеет дело.

%Более того, подход Сирла применим к любому набору детерминированных преобразований текста. 
Математически идею Сирла можно сформулировать как указание на различие между синтаксисом языка и его семантикой (смыслом). Сирл считает, что понимания синтаксических правил может оказаться достаточно для <<осмысленных>> преобразований текста, но не гарантирует понимания семантики. Машина Тьюринга оперирует только синтаксисом, а значит принципиально не может прийти к семантическим преобразованиям.

Наиболее известным контраргументом к примеру Сирла является {\it системный подход}: человек в комнате не знает китайского языка, но система, состоящая из комнаты, человека и книги правил, знает китайский язык.

Сирл возражает, что принципиально возможно выучить книгу наизусть и имитировать поведение человека, говорящего на китайском, по-прежнему не понимая этого языка.

Дэниель Денетт продолжает развитие системного подхода и отмечает:
%что классическая машина Тьюринга вряд ли может пройти тест.
% Если 
если что-то и сможет пройти его, это будет гигантский параллельный вычислитель, с программой соответствующей сложности. И выучить книгу правил будет не так просто, как полагает Сирл.

Известен также другой контраргумент к идее Сирла, происходящий из робототехники. Он заключается в том, что человек заранее предупрежден относительно неспособности компьютера к мышлению. И это искажает его суждения. В то же время, если некий робот в человеческом обличии и с соответствующей программой ходил бы по улицам и общался с окружающими, никто не усомнился бы в том, что он понимает китайский язык.

\section{Другие тесты машинного интеллекта}

Все известные тесты, определяющие наличие у машины мышления, так или иначе перерабатывают идею Алана Тьюринга. Модификации теста Тьюринга можно условно разделить на три группы: упрощения, расширения, усложнения.

\subsection{Упрощение теста}

Первая группа включает модификации, которые стремятся ослабить тест настолько, чтобы он соответствовал действительному положению дел в теории искусственного интеллекта.

Интеллектуальный тест {\it минимального сигнала} накладывает ограничение на формат разговора. В этом тесте судья задает лишь вопросы, на которые машина отвечает <<да>> или <<нет>>. Вопросы могут быть как на логическое мышление, так и на культурно-исторические темы. Также могут быть неформальные вопросы, например, <<правда ли, что солнце больше яблока.>> Такой тест проще, чем Тест Тьюринга, но только с точки зрения правил проведения. С точки зрения прохождения, этот тест не выглядит более простым.

Другое упрощение теста Тьюринга — это {\it экспертный тест.} Предполагается, что машина должна давать формальные ответы на вопросы из ограниченной предметной области. Иллюстрацией к такому тесту может являться <<искусственный интеллект космического корабля.>> Такой интеллект способен распознать речь и поддержать разговор в области, касающейся состояния корабля или управления им, но не может выйти за рамки данной предметной области. Экспертный Тест Тьюринга проще классического, так как не включает в себя абстрактных и чувственных понятий; кроме того он зачастую более полезен на практике. 

Современные экспертные системы успешно проходят такой тест. Как правило в них используется искусственный (формализованный) язык общения. Экспертные системы уже сейчас приносят пользу, и во многих областях даже превосходят живых экспертов.

Начиная с 1991 года регулярно проходят соревнования на приз Любнера. В этих соревнованиях принимают участие компьютерные программы, задача которых --- убедить жюри в том, что они обладают мышлением. С 1995 года в этих соревнованиях появилась номинация по сокращённому тесту Тьюринга. 

Согласно правилам соревнований на приз Любнера, участнику может быть задано всего четыре типа вопросов:
\begin{itemize}
  \item Вопросы о времени: сколько сейчас времени, утро сейчас или вечер, какой по счёту идёт раунд, и т. д.
  \item Вопросы о вещах: что такое молоток, для чего он предназначен, для чего нужно такси, и т. д.
  \item Вопросы о соотношении вещей: что больше, апельсин или грейпфрут, что быстрее, самолёт или поезд, и т. д.
  \item Вопросы на память. Дается предложение: <<У меня есть друг Генри, играющий в теннис>> --- и через некоторое время вопрос: <<В какую игру любит играть мой друг Генри?>>
\end{itemize}

Видно, что такие вопросы не предполагают наличия нетривиального интеллекта у отвечающего. Однако, ни 18 лет назад, ни сейчас этим простым требованиям не способен удовлетворить ни один искусственный интеллект.

В целом можно сказать, что идеи {\it нового} теста Тьюринга чаще всего развивались именно в сторону упрощения. Эту тенденцию можно объяснить с точки зрения прикладных задач. Однако на сегодняшний день нет теста, который был бы признан научным сообществом.

\subsection{Расширение теста}

Расширение теста Тьюринга предполагает увеличение предметной области, которую охватывает тест. 

Гундерсон \cite{gunderson_1985} указал на то, что Тест Тьюринга требует от машины демонстрации лишь некоторой интеллектуальной деятельности, и не гарантирует успех применения такой машины в прикладных областях. Он предложил расширить Тест Тьюринга требованием, чтобы машина проявляла другие когнитивные качества, а также определённый набор знаний. С точки зрения Гундерсона, подобный подход позволил бы гарантировать, что интеллект, прошедший модифицированный Тест Тьюринга, пригоден в прикладных целях.

На самом деле, идея Гундерсона --- это скорее набор дополнительных рекомендаций к исходному тесту Тьюринга, а не его коренное изменение: в процессе тестирования всегда можно попросить участника продемонстрировать когнитивные качества и специальные знания. 

Принципиально другой подход предложил Баресси \cite{barresi_1987}. Он считает, что Тест Тьюринга ограничен, так как требует от участника лишь умения вводить в заблуждение. А значит, такой тест не даёт гарантию целостности мышления. 

Баресси предлагает новый тест, идею и название которого он находит в произведениях фантаста Станислава Лема. Тест получил название в честь цикла произведений Кибериада. Этот тест направлен на проверку человечности интеллекта кибера (робота).

Тест рассчитан не на отдельного представителя, а на социум. Тест считается пройденным, лишь если киберы в состоянии постепенно заместить людей в человеческом сообществе. При этом общество должно сохранить структуру и продолжить своё развитие. Продолжительность теста — несколько миллионов лет.

Фактически Баресси предлагает на роль жюри в тесте Тьюринга пригласить саму Природу.

Из данного теста Баресси выводит несколько интересных идей. Одна из них заключается в следующем: поскольку сам тест занимает значительное время, его должен предварять экзамен. Индивид решивший пройти тест Кибериады должен сначала пройти экзамен, на котором ему необходимо доказать свою способность пройти сам тест. Экзамен --- тот же Тест Тьюринга --- где комиссия будет принимать ответы по тем или иным аспектам жизни социума. Продолжительность экзамена может достигать несколько сотен лет.

\subsection{Усложнение теста}

В случае усложнения, Тест Тьюринга рассматривается не как прикладной критерий, а лишь как абстрактный {\it истинный} тест.

Харнад \cite{harnand_1991} предложил изменить Тест Тьюринга, указав на необходимость полного обмена информацией с искусственным интеллектом. Фактически, речь идёт о построении робота с моторными, зрительными, аудиальными и тактильными функциями. В настоящее время возможность прохождения такого теста целиком и полностью относится к области научной фантастики.

{\it Тест Лавлэйс} был предложен Брингйорном \cite{bringsjord_1992}, как продолжение идеи возражения леди Лавлэйс. Суть в том, что, по Брингйорну, истинно интеллектуальной является система, которая способна выдавать разумные ответы на входные данные. При этом ответы не должен быть вызваны ни аппаратными, ни программными ошибками; и в это же время ни один человек не должен быть способен объяснить, исходя из архитектурных данных и анализа программы, каким образом был получен этот ответ.

Такая постановка достаточно сомнительна, потому что в случае длинной и специально запутанной программы, ни один человек не будет способен понять её за адекватное время, тогда как алгоритм, скрывающийся за ней, может быть очень простым. Кроме того, не известна степень детализации, с которой должно быть проведено объяснение программы.

Наконец, подход Швайцера комбинирует подходы Харнада и Баресси. Швайцер настаивает на том, что настоящую оценку интеллекта можно получить только в случае, когда тело робота обладает схожими с человеческими способностями, а оценка такого интеллекта может быть получена лишь за исторический срок. То есть, Швайцер предлагает создать колонию человекоподобных роботов и пронаблюдать их развитие.

\subsection{Обратный тест Тьюринга}

Стюарт Ватт в 1996 году предложил идею обратного теста Тьюринга. Машина  проходит обратный Тест Тьюринга, если:
\begin{itemize}
  \item задавая вопросы в классическом тесте она не в состоянии отличить случай, когда она играет против двух людей, от случая, когда она играет против человека и машины прошедшей Тест Тьюринга;
  \item задавая вопросы в классическом тесте она отличает человека от машины не прошедшей Тест Тьюринга.
\end{itemize}

Данный тест вынесен в отдельную категорию, поскольку нельзя однозначно определить он усложняет или расширяет исходный тест. Существует несколько различных взглядов по этому вопросу \cite{sayagin_2000}.

\section{Хронология развития представлений об \\ искусственном интеллекте}

Приведённые выше идеи являются характерными для 80-х годов 20-го века. С развитием компьютерных наук и теории искусственного интеллекта стало ясно, что тесты тьюринговского типа, полагающие тождественность интеллекта и подражания человеку, оказываются неудобными. 

Уайтби \cite{whitby_1996} указывает основные этапы в развитии представлений об искусственном интеллекте:
\begin{itemize}
  \item 1950-1966гг. Тест Тьюринга в это время является источником вдохновения для всех, кто связан с искусственным интеллектом.
  \item 1966-1973гг. Всеобщее разочарование, связанное с тем, что наиболее многообещающие пути развития не дали ожидаемого результата.
  \item 1973-1990гг. Тест Тьюринга полностью разочаровал философов, но ещё волнует умы разработчиков искусственного интеллекта.
  \item 1990г. Тест Тьюринга предан истории.
\end{itemize}

Основной причиной последнего факта является отсутствие видимого прогресса в прохождении теста Тьюринга. До сих пор нет ни одной системы, которая хотя бы условно может считаться прошедшей Тест Тьюринга. При этом, Тест Тьюринга долгое время являлся единственным критерием интеллектуальности. Отсутствие успехов привело к тому, что сейчас невозможно сравнить различные интеллектуальные системы.

Есть и более глубокие причины угасания интереса к тесту Тьюринга. В настоящее время популярна идея, что если даже какая-то программа сможет пройти Тест Тьюринга, то она окажется совершенно неприменимой на практике: для управления производством, прогнозирования, решения прикладных задач и так далее. В самом деле, если машина не отличается от человека, это означает что она обладает всеми его недостатками. Например, леностью, забывчивостью и способностью ошибаться. 

Может показаться, что Тест Тьюринга был рассчитан на создание общих методов в области искусственного интеллекта, прохождение этого теста привело бы к решению многих прикладных задач. Однако в настоящее время, для решения прикладных задач и в чат ботах (на соревнованиях Любнера) используются совершенно разные технологии. Решение прикладных задач не коррелирует с прохождением классического теста Тьюринга. 

Оказалось, что в практических задачах нужен интеллект, похожий на человеческий, но отличающийся рядом характеристик. В результате, отмечает Уитби, исследование проблемы искусственного интеллекта пошло по странному пути: интеллект, прошедший Тест Тьюринга, на самом деле никому не нужен.

Уитби фактически проводит аналогию между проблемой искусственного интеллекта и проблемой полётов. Первые исследователи стремились изучать птиц и пытались создать механизмы, подражающие им. Но настоящего успеха удалось добиться совсем в другой области, а именно через более детальное исследование аэродинамики. Аэродинамика объяснила, почему птицы могут летать, и почему люди не смогут летать как птицы. Это позволило создать летающие машины, основанные на других принципах. Если бы исследователи продолжали работать только над подражанием птичьему полёту, возможно самолёт и появился бы, но значительно позже.

Забвение теста Тьюринга, как единого теста для систем искусственного интеллекта, привело к тому, что единая прежде наука распалась на множество независимых течений. <<Системы искусственного интеллекта>> были замещёны <<интеллектуальными системами.>> Новые системы не способны пройти классический Тест Тьюринга, поскольку в них не используются естественные языки. В отличие от прежних, эти системы не воспринимаются как самоцель, а лишь как средство для решения прикладных задач. Соответственно, разные течения находят применение при решении раных классов задач.

Представляется вполне вероятным, что со временем, сделав очередной виток, наука ещё вернётся к исследованию классического теста Тьюринга.


\begin{thebibliography}{99}
\bibitem{turing_1936}
Turing, A. On computable numbers, with an application to the entscheidungsproblem / A. Turing // Proceedings of the London Mathematical Society, Series 2 — 1936 — Vol. 42 — Pp. 233-265.

\bibitem{church_1932}
Church, A. A set of postulates for the foundation of logic / A. Church // Annals of mathematics, secod series — 1932 — Vol. 33, no. 2 — Pp. 346-366.

\bibitem{kleene_1936}
Kleene, S. C. A theory of positive integers in formal logic / S. C. Kleene // American Journal of Mathematics — 1936 — Vol. 57 — Pp. 219-244.

\bibitem{markov_1984}
Марков, А. А. Теория алгоритмов / А. А. Марков, Н. М. Нагорный — Наука, 1984.

\bibitem{turing_1950}
Turing A. Computing Machinery and Intelligence / A. Turing // Mind — 1950 — Vol. LIX, no. 236 — 433-460.

\bibitem{sayagin_2000}
Saygin A. P., Ciceckli I., Akman V. Turing Test: 50 Years Later /  A. P. Saygin,  I. Ciceckli, V. Akman // Minds and Machines — Vol. 10 — 2000 — 463-518.

\bibitem{bringsjord_1992}
Bringsjord, S. What robots Can and Can't Be / S. Bringsjord — Kluwer — 1992.

\bibitem{block_1981}
Block, N. Psychilogism and Behaviorism / Ned Block // The Philosophical Review — Vol. LXXXX, No. 1 — January 1981 — 5-43.

\bibitem{searle_1980}
John R. Searle. Is the Brain's Mind a Computer Program? /  John R. Searle — 1980.

\bibitem{gunderson_1985}
Gunderson, K. Mentality and Machines / K. Gunderson // University of Minnisonta Press, 1985.

\bibitem{barresi_1987}
Barresi, J. Prospects for the Cyberiad: Certain Limits on Human Self-Knowledge in the Cybernetic Age / John Barres // Journal for the Theory of Social Behaviour — 17:1 — March 1987.

\bibitem{harnand_1991}
Harnand, S. Other bodies, other minds: A machine incarnation of an old hilosophical problem / S. Harnand // Minds and Machines — 1991 — Vol.~1.

\bibitem{whitby_1996}
Whitby, B. R. The turing test: Ai's biggest blind alley? / B. R. Whitby // Machines and Thought: The Legacy of Alan Turing. Mind Association Occasional Series — Oxford University Press — Vol. 1 — 1996.

\bibitem{akman}
Varol Akman, Patric Blackburn, Editorial: Alan Turing and Artifical Intelligenc.

\bibitem{okulovsky_2008}
Ю. С. Окуловский, Системы искусственного интеллекта,УрГУ 2008.

\bibitem{lem_1995}
Станислав Лем, Тайна китайской команты 1995. Перевод. Язневич В.И. 2003.

\bibitem{bogatiryov}
Руслан Богатырев, Компьютерные шахматы. Анатомия искусственного интеллекта.

\bibitem{lifshic}
Юрий Лифшиц Михайлович, Принципы развития теории алгоритмов.

\end{thebibliography}

\end{document}
