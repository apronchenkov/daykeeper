% 2009-01-11

\documentclass[draft]{article}
%%
% Пакет позволяющий определить, что используется: latex или pdflatex?
%%
\usepackage{ifpdf}

%%
% Пакеты AMS\cdot
%%
\usepackage{amsfonts, amsmath, amsthm}

%%
% Определяем, используется ли XeTex?
%%
\ifx\XeTeXversion\undefined
  %%
  % Набор пакетов для работы с графическими файлами
  %%
  \ifpdf
    \usepackage[pdftex]{graphicx}
    \usepackage{cmap}
  \else
    \usepackage{graphicx}
  \fi

  \usepackage[utf8x]{inputenc} 

\else
  %%
  % Набор пакетов для поддержки русского языка в XeLaTeX
  %%
  \usepackage[cm-default]{fontspec}
  \usepackage{xunicode}
  \usepackage{xecyr}

  % Setting default fonts
  \usepackage{unicode-math}
  \setmainfont[Mapping=tex-text]{Cambria}
  \setmathfont{Cambria Math}
  
  %\setmainfont{Times New Roman} 
  %\setmainfont{Georgia}
  %\setmainfont{Lucida Console}
\fi

%%
% Дополнительные настройки
%%
\usepackage[english,russian]{babel} 
\usepackage{indentfirst}

\newtheorem{define}{Определение}
\newtheorem{theorem}{Теорема}
\newtheorem{lemma}{Лемма}
\newtheorem{remark}{Замечание}

\newcommand{\abs}[1]{\lvert{}#1{}\rvert}
\newcommand{\Abs}[1]{\lVert{}#1{}\rVert}
\newcommand{\ak}[1][k]{\alpha^{(#1)}}
\newcommand{\minak}[1][k]{\underline{\alpha}^{(#1)}}
\newcommand{\maxak}[1][k]{\overline{\alpha}^{(#1)}}
\newcommand{\mk}[1][k]{m^{(#1)}}
\newcommand{\Mk}[1][k]{M^{(#1)}}
\newcommand{\tk}[1][k]{t^{(#1)}}
\newcommand{\Tk}[1][k]{T^{(#1)}}

%%
% Начало документа
%%
\begin{document}

{\em Оценка размера окна.}
Рассмотрим устоявшуюся систему в который:
\begin{itemize}
  {\item все пакеты доставляются мгновенно;}
  {\item каждый пакет имеет размер 1300;}
  {\item каждый двадцатый пакет теряется.}
\end{itemize}

Пусть $w_{min}$ -- минимальный размер окна возникающий в устоявшейся системе. Тогда,
\[ (w_{min} + \frac{1300 \cdot 1024}{w_{min}} \cdot 19) \cdot 0.78 = w_{min}. \]
Отсюда мы можем выразить $w_{min} = 10723.$ Соответственно $w_{max}$ будет
\[ w_{max} = w_{min} + \frac{1300 \cdot 1024}{w_{min}} \cdot 19 = 13082. \]
Среднее теоретическое значение будет $w = 11902.$

{\em Оценка пропускной сопособность.} Если теперь предположить, что $rtt = 0.3$, тогда, теоретически, средняя пропусная сособность будет:
\[ b = \frac{w}{rtt} = 36068. \]

\end{document}
