% 2009-01-14
% 2009-01-11
% 2009-01-10

\documentclass{article}
%%
% Пакет позволяющий определить, что используется: latex или pdflatex?
%%
\usepackage{ifpdf}

%%
% Пакеты AMS*
%%
\usepackage{amsfonts, amsmath, amsthm}

%%
% Определяем, используется ли XeTex?
%%
\ifx\XeTeXversion\undefined
  %%
  % Набор пакетов для работы с графическими файлами
  %%
  \ifpdf
    \usepackage[pdftex]{graphicx}
    \usepackage{cmap}
  \else
    \usepackage{graphicx}
  \fi

  \usepackage[utf8x]{inputenc} 

\else
  %%
  % Набор пакетов для поддержки русского языка в XeLaTeX
  %%
  \usepackage[cm-default]{fontspec}
  \usepackage{xunicode}
  \usepackage{xecyr}

  % Setting default fonts
  \usepackage{unicode-math}
  \setmainfont[Mapping=tex-text]{Cambria}
  \setmathfont{Cambria Math}
  
  %\setmainfont{Times New Roman} 
  %\setmainfont{Georgia}
  %\setmainfont{Lucida Console}
\fi

%%
% Дополнительные настройки
%%
\usepackage[english,russian]{babel} 
\usepackage{indentfirst}

\newtheorem{define}{Определение}
\newtheorem{theorem}{Теорема}
\newtheorem{lemma}{Лемма}
\newtheorem{remark}{Замечание}

\newcommand{\abs}[1]{\lvert{}#1{}\rvert}
\newcommand{\Abs}[1]{\lVert{}#1{}\rVert}
\newcommand{\ak}[1][k]{\alpha^{(#1)}}
\newcommand{\minak}[1][k]{\underline{\alpha}^{(#1)}}
\newcommand{\maxak}[1][k]{\overline{\alpha}^{(#1)}}
\newcommand{\mk}[1][k]{m^{(#1)}}
\newcommand{\Mk}[1][k]{M^{(#1)}}
\newcommand{\tk}[1][k]{t^{(#1)}}
\newcommand{\Tk}[1][k]{T^{(#1)}}

%%
% Начало документа
%%
\begin{document}

\section{Постановка задачи}

Введём вспомогательные обозначения. Пусть $A = \left( a_{ij} \right)_{n \times m}$ — произвольная матрица, $S$ —
 произвольное множество, тогда
\[
  A_{i*} = \{ a_{i1},\ a_{i2},\ \ldots,\ a_{im} \} , \\
\]
\[
  A_{*j} = \{ a_{1j},\ a_{2j},\ \ldots,\ a_{nj} \} , \\
\]
\[
  \Abs{S} = \sum_{s \in S}{s}.
\]

Даны последовательности матриц $\left\{ \Mk \right\}$ и $\left\{ \Tk \right\}$;
\[ \Mk = \left( \mk_{ij} \right)_{n\times{}n}, \]

\begin{displaymath}
  \ak_{i} = \left\{
  \begin{array}{cl}
    1, & \quad \text{если} \ i = n - 1 \\
    \frac{\Abs{\Mk_{*i}}}{\Abs{\Mk_{i*}}}, & \quad \text{если} \ i \not= n - 1 \\
  \end{array}
  \right. ,
\end{displaymath}

\begin{displaymath}
  \Tk = \left(
  \begin{array}{cccc}
     \ak_1 &      0 & \ldots &      0 \\
         0 &  \ak_2 & \ldots &      0 \\
    \vdots & \vdots & \ddots & \vdots \\
         0 &      0 & \ldots &  \ak_n \\
  \end{array}
  \right)
\end{displaymath}

Последовательность $\Mk$ определяется начальной матрицей $\Mk[0]$ и правилом перехода:
\[ \Mk[k + 1] = \Tk \Mk. \]

Сформулируем задачу. Пусть $\Mk[0]$ — положительно определённая матрица. Необходимо показать, что $\Tk \rightarrow E$ при $k \rightarrow +\infty$.


\section{Начальные условия}

\begin{define}
Матрица $A$ {\it называется положительно определённой,} если для любого ненулевого вектора-строки $x$ справедливо $x A x^T > 0.$
\end{define}

Например следующая матрица положительно определённая:  
\begin{displaymath}
  A = \left(
  \begin{array}{ccc}
      1 & 0 & -1 \\
      0 & 1 &  0 \\
      0 & 0 &  1 \\
  \end{array}
  \right).
\end{displaymath}

\begin{remark}
  \label{remark:row_sum_zero}
  Если у матрицы $\Mk$ есть строка, сумма элементов которой равна нулю, то к такой матрице невозможно применить правило перехода — возникает деление на ноль.
\end{remark}

\begin{remark}
  \label{remark:col_sum_zero}
  Если у матрицы $\Mk$ есть столбец, сумма элементов которого равна нулю, то у матрицы $\Mk[k + 1]$ имеется нулевая строка.
\end{remark}

Согласно замечаниям \ref{remark:row_sum_zero} и \ref{remark:col_sum_zero} положительной определённости для начальных условий не достаточно. Существуют примеры в которых последовательность оказывается конечной.

Введём дополнительное ограничение: пусть $\Abs{\Mk[0]_{i*}} \not= 0$ и $\Abs{\Mk[0]_{*j}} \not= 0$ для всех $i, j \in \left\{ 1, 2, \ldots, n \right\}.$

Рассмотрим пример:
\begin{gather*}
  \Mk[0] = \left(
  \begin{array}{ccc}
      1 & 0 & -2 \\
      0 & 1 &  0 \\
      0 & 0 &  3 \\
  \end{array}
  \right), \\
  \Mk[1] = \left(
  \begin{array}{ccc}
     -1 & 0 &  2 \\
      0 & 1 &  0 \\
      0 & 0 &  1 \\
  \end{array}
  \right), \\
  \Mk[2] = \left(
  \begin{array}{ccc}
      1 & 0 & -2 \\
      0 & 1 &  0 \\
      0 & 0 &  3 \\
  \end{array}
  \right), \\
  \ldots
\end{gather*}
В этой последовательности начальная матрица положительно определённая, и сумма элементов любой её строки (столбца) не ноль.
Последовательность не имеет предела, так как у неё есть период. Это произошло из-за отрицательной суммы элементов первой строки.

Ещё один пример:
\begin{gather*}
  \Mk[0] = \left(
  \begin{array}{ccc}
     1 &            0 & -\frac{1}{2} \\
     0 &            1 &            0 \\
     0 & -\frac{1}{2} &            1 \\
  \end{array}
  \right), \\
  \Mk[1] = \left(
  \begin{array}{ccc}
     2 &            0 & -1 \\
     0 &            1 &  0 \\
     0 & -\frac{1}{2} &  1 \\
  \end{array}
  \right).
\end{gather*}
Начальная матрица вновь положительно определённая. В этом примере у матрицы $\Mk[1]$ сумма элементов в последнем столбце равна нулю. Согласно замечаниям \ref{remark:row_sum_zero} и \ref{remark:col_sum_zero} мы получаем конечную последовательность.

Таким образом указанных ограничений явно не достаточно, чтобы предел последовательности существовал.


В данной работе, мы будем использовать другое начальное условие: 
\begin{equation}
  \label{eq:constraint}
  \mk[0]_{ij} > 0 \quad \text{для всех} \quad i,j \in \left\{ 1, 2, \ldots n \right\}
\end{equation}
— ограничение положительной определённости матрицы снимается.


Отметим, что если разрешить $\mk[0]_{ij}$ быть нулями, то возможен пример когда предел последовательностей существует, но отличен от ожидаемого:
\begin{displaymath}
  \Mk[0] = \left(
  \begin{array}{ccc}
      2 & 1 & 0 \\
      0 & 1 & 0 \\
      0 & 1 & 2 \\
  \end{array}
  \right);
\end{displaymath}
\begin{displaymath}
  \Mk \rightarrow \left(
  \begin{array}{ccc}
      0 & 0 & 0 \\
      0 & 1 & 0 \\
      0 & 0 & 0 \\
  \end{array}
  \right)
  \quad \text{при} \quad k \rightarrow +\infty,
\end{displaymath}
\begin{displaymath}
  \Tk \equiv \left(
  \begin{array}{ccc}
      \frac{2}{3} & 0 &           0 \\
                0 & 1 &           0 \\
                0 & 0 & \frac{2}{3} \\
  \end{array}
  \right).
\end{displaymath}


\section{Вспомогательные построения}

Обозначим $\maxak = \max \{ \ak_i \},$ $\minak = \min \{ \ak_i \}.$

\begin{lemma}
  \label{lemma:constraint}
  Для любого $i, j$ и $k$ верно: $\mk_{ij} > 0$ и $\ak_i > 0. $
  \hfill$\square$
\end{lemma}

\begin{lemma}
  \label{lemma:one}
  Для любого $k$ верно: $\minak \leq 1 \leq \maxak$.
  \hfill$\square$
\end{lemma}

\begin{lemma}
  \label{lemma:ak_next}
  Для любых $i \not= n - 1$ и $k$ верно:
  \[ \ak[k + 1]_i = \frac{\sum_{j = 1}^n{\ak_j\mk_{ji}}}{\sum_{j = 1}^n{\mk_{ji}}}. \]
\end{lemma}
\begin{proof}
  Распишем значение $\ak[k + 1]_i$ исходя из определения:
  \[ \ak[k + 1]_i = \frac{\Abs{\Mk[k + 1]_{*i}}}{\Abs{\Mk[k + 1]_{i*}}} = 
                    \frac{\Abs{(\Tk\Mk)_{*i}}}{\Abs{(\Tk\Mk)_{i*}}}.
  \]
  Распишем числитель:
  \[ \Abs{(\Tk\Mk)_{*i}} = \sum_{j = 1}^n{\ak_j\mk_{ji}}. \]
  Распишем знаменатель:
  \begin{eqnarray*}
    \Abs{(\Tk\Mk)_{i*}} & = & \sum_{j = 1}^n{\ak_i\mk_{ij}} = \ak_i \sum_{j = 1}^n{\mk_{ij}} = \\
                        & = & \ak_i \Abs{\Mk_{i*}} = \frac{\Abs{\Mk_{*i}}}{\Abs{\Mk_{i*}}} \Abs{\Mk_{i*}} \\
                        & = & \Abs{\Mk_{*i}} = \sum_{j = 1}^n{\mk_{ji}}. 
  \end{eqnarray*}

  Чтобы получить искомое утверждение, осталось подставить результаты в исходное уравнение.
\end{proof}

\begin{lemma}
  \label{lemma:subsegment}
  Для любого $k$ верно: $\minak \leq \minak[k + 1] \leq \maxak[k + 1] \leq \maxak[k]$.
\end{lemma}
\begin{proof}
  Покажем, что для произвольного $i$ верно неравенство
  \begin{equation}
    \label{lemma:subsegment:eq:1}
    \minak \leq \ak[k + 1]_i \leq \maxak.
  \end{equation}

  В случае $i = n - 1$ неравенство (\ref{lemma:subsegment:eq:1}) очевидно верно.

  Пусть $i \not= n - 1$. Тогда по определению
  \[ \minak \leq \ak_i \leq \maxak, \]
  и следовательно (используя результаты лемм \ref{lemma:constraint} и \ref{lemma:ak_next})
  \begin{gather*} 
     \minak = \minak\frac{\sum_{j = 1}^n{\mk_{ji}}}{\sum_{j = 1}^n{\mk_{ji}}} = \frac{\sum_{j = 1}^n{\minak\mk_{ji}}}{\sum_{j = 1}^n{\mk_{ji}}} \leq \\
     \leq \ak[k + 1]_i \leq \\
     \leq \frac{\sum_{j = 1}^n{\maxak\mk_{ji}}}{\sum_{j = 1}^n{\mk_{ji}}} = \maxak\frac{\sum_{j = 1}^n{\mk_{ji}}}{\sum_{j = 1}^n{\mk_{ji}}} = \maxak.
  \end{gather*}

  Мы доказали неравенство (\ref{lemma:subsegment:eq:1}). Из него следует утверждение леммы.
\end{proof}

\begin{lemma}
  \label{lemma:final}
  Если $\minak = \minak[k + 1]$, то $\ak_i = 1,$ для всех $i \in \left\{ 1, 2, \ldots, n \right\}.$
\end{lemma}
\begin{proof}
  Пусть утверждение леммы верно, и $i \not= n - 1$ такое, что
  \[ \ak[k + 1]_i = \minak[k + 1] = \minak. \]
  Согласно лемме \ref{lemma:ak_next}:
  \[ \frac{\sum_{j = 1}^n{\ak_j\mk_{ji}}}{\sum_{j = 1}^n{\mk_{ji}}} = \ak[k + 1]_i = \minak = \frac{\sum_{j = 1}^n{\minak\mk_{ji}}}{\sum_{j = 1}^n{\mk_{ji}}}, \]
  или
  \[ \sum_{j = 1}^n{\ak_j\mk_{ji}} = \sum_{j = 1}^n{\minak\mk_{ji}}. \]

  По определению мы имеем $\ak_j \geq \minak$, кроме того из леммы \ref{lemma:constraint} мы знаем, что $\mk_{ji} > 0$. Значит
  \[ \minak = \ak_j \quad \text{для всех} \quad j \in \left\{ 1, 2, \ldots, n \right\} \]
  и
  \[ \minak = \maxak. \]

  Из леммы \ref{lemma:one} следует, что $\minak = \maxak = 1$.
\end{proof}


\section{Теорема}

\begin{theorem}
  Если выполняется начальное условие (\ref{eq:constraint}), тогда последовательность матриц $\Tk$ поэлементно сходится к единичной матрице.
  % при $k \rightarrow +\infty.$
\end{theorem}
\begin{proof}
  Из лемм \ref{lemma:subsegment} и \ref{lemma:final} следует, что последовательность 
  \[ \left\{ [ \minak, \maxak ] \right\} \]
  образует систему вложенных отрезков с единственной общей точкой — $1$.
%  Используя математический анализ, отсюда легко получаем, что
  Отсюда легко получаем, что
  $\ak_i \rightarrow 1$ для всех $i.$
%  $\ak_i \rightarrow 1$ для любого $i \in \left\{ 1, 2, \ldots, n \right\}.$

%  \[ \ak_i \rightarrow 1, \qquad \text{для всех $i \in \left\{ 1, 2, \ldots, n \right\}$} . \]
%  По лемме о вложенных отрезка (стандартный курс математического анализа) очевидно получаем $\ak_i \rightarrow 1$ для всех $i.$
\end{proof}

\end{document}
