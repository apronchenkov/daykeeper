\documentclass[draft]{article}
%%
% Пакет позволяющий определить, что используется: latex или pdflatex?
%%
\usepackage{ifpdf}

%%
% Пакеты AMS*
%%
\usepackage{amsfonts, amsmath, amsthm}

%%
% Определяем, используется ли XeTex?
%%
\ifx\XeTeXversion\undefined
  %%
  % Набор пакетов для работы с графическими файлами
  %%
  \ifpdf
    \usepackage[pdftex]{graphicx}
    \usepackage{cmap}
  \else
    \usepackage{graphicx}
  \fi

  \usepackage[utf8x]{inputenc} 

\else
  %%
  % Набор пакетов для поддержки русского языка в XeLaTeX
  %%
  \usepackage[cm-default]{fontspec}
  \usepackage{xunicode}
  \usepackage{xecyr}

  % Setting default fonts
  \usepackage{unicode-math}
  \setmainfont[Mapping=tex-text]{Cambria}
  \setmathfont{Cambria Math}
  
  %\setmainfont{Times New Roman} 
  %\setmainfont{Georgia}
  %\setmainfont{Lucida Console}
\fi

%%
% Дополнительные настройки
%%
\usepackage[english,russian]{babel} 
\usepackage{indentfirst}

\DeclareMathOperator{\slerp}{slepr}
\DeclareMathOperator{\squad}{squad}

%%
% Начало документа
%%
\begin{document}

\title{Алгебра кватернионов и их дифференциальное исчисление.}
\author{David Eberly\footnote{перевод Александра Геннадьевича Пронченкова}}
\maketitle

В данном тексте приведены краткие сведения об алгебре кватернионов и их дифференциальном исчислении, а так же о связи кватернионов и преобразований пространства. Материал представленный здесь основан на статье \cite{KenShoemake}.

\section{Алгебра кватернионов}

Кватернион является вектором
%определяется как вектор 
в четырёхмерном пространстве $q = w + x i + y j + z k$, где $w$, $x$, $y$ и $z$ действительные числа, а $1, i, j$ и $k$ базис этого пространства. 

Определим некоторые операции на множестве кватернионов. Пусть $q_n = w_n + x_n i + y_n j + z_n k,\ n \in \{0, 1\}$ два кватерниона; сложение и вычитание действуют следующим образом: 
\begin{multline}
\label{eq:1}
q_0 \pm q_1 = (w_0 + x_0 i + y_0 j + z_0 k) \pm (w_1 + x_1 i + y_1 j  +z_1 k) = {} \\
{} = (w_0 \pm w_1) + (x_0 \pm x_1) i + (y_0 \pm y_1) j + (z_0 \pm z_1) k .
\end{multline}

Определим операцию умножение для элементов базиса $i$, $j$ и $k$: $i^2 = j^2 = k^2 = -1$, $i j = -j i = k$, $j k = - k j = i$ и $k i = - i k = j$ — этого достаточно чтобы определить операцию умножение для произвольных кватернионов:
\begin{multline}
\label{eq:2}
q_0 q_1 = (w_0 + x_0 i + y_0 j + z_0 k)(w_1 + x_1 i + y_1 j + z_1 k) \\
= (w_0 w_1 - x_0 x_1 - y_0 y_1 - z_0 z_1) + (w_0 x_1 + x_0 w_1 + y_0 z_1 - z_0 y_1) i + {} \\
{} + (w_0 y_1 - x_0 z_1 + y_0 w_1 + z_0 x_1) j + (w_0 z_1 + x_0 y_1 - y_0 x_1 + z_0 w_1) k .
\end{multline}
Умножение кватернионов не коммутативно, то есть произведение $q_0 q_1$ не обязательно равно произведению $q_1 q_0$.

Сопряжённым кватернионом к $q$ называется кватернион
\begin{equation}
\label{eq:3}
q^* = (w + x i + y j + z k)^* = w - x i - y j - z k .
\end{equation}
Сопряжение обладает следующими свойствами: $(q^*)^* = q$ и $(p q)^* = q^* p^*$.

Нормой кватерниона $q$ называется величина:
\begin{equation}
\label{eq:4}
N(q) = N(w + xi + yj + zk) = w^2 + x^2 + y^2 + z^2 .
\end{equation}
Норма обладает следующими свойствами: $N(q) = q  q^*$ и $N(p q) = N(p) N(q)$.

Кватернион обратный по умножению к $q$ обозначается $q^{-1}$
% Обратный по умножению кватернион для $q$ обозначается $q^{-1}$
и обладает свойством $q q^{-1} = q^{-1} q = 1$. Он устроен следующим образом:
\begin{equation}
\label{eq:5}
q^{-1}=q^*/N(q) ,
\end{equation}
где деление кватерниона на скаляр это простое покомпонентное деление. Операция взятия обратного кватерниона обладает следующими свойствами: $(p^{-1})^{-1} = p$ и $(p q)^{-1} = q^{-1} p^{-1}$.

Определим простую, но полезную функцию — функцию выбора:
\begin{equation}
\label{eq:6}
W(q) = W(w + x i + y j + z k) = w , 
\end{equation}
которая выбирает {\it действительную часть} кватерниона. Эта функция обладает свойством: $W(q) = (q + q^*)/2$.

Кватернион $q$ можно рассмотреть как $q = w + \widehat{v}$, где $\widehat{v} = x i + y j + z k$. Если представлять $\widehat{v}$ как трёхмерный вектор $(x, y, z)$, то произведение кватернионов можно переписать с использованием скалярного ($\bullet$) и векторного ($\times$) произведений:
\begin{equation}
\label{eq:7}
(w_0 + \widehat{v}_0)(w_1 + \widehat{v}_1) = (w_0 w_1 - \widehat{v}_0\bullet\widehat{v}_1) + w_0 \widehat{v}_1 + w_1\widehat{v}_0 + \widehat{v}_0\times\widehat{v}_1 .
\end{equation}
При этом становится очевидным, что $q_0 q_1 = q_1 q_0$ тогда и только тогда, когда $\widehat{v}_0\times\widehat{v}_1 = 0$ (то есть, вектора $\widehat{v}_0$ и $\widehat{v}_1$ параллельные).

Кватернион $q$ может быть рассмотрен как четырёхмерный вектор $(w, x,$ $y, z)$. При этом возникает операция скалярное умножение для кватернионов:
\begin{equation}
\label{eq:8}
q_0\bullet q_1 = w_0 w_1 + x_0 x_1 + y_0 y_1 + z_0 z_1 = W(q_0 q_1^*) .
\end{equation}

Единичный кватернион — это кватернион с единичной нормой. Кватернион обратный для единичного кватерниона и произведение двух единичных кватернионов так же будут единичными кватернионами. Единичный кватернион $q$ можно записать в следующем виде: 
\begin{equation}
q = \cos\theta  + \widehat{u}\sin\theta ,
\end{equation}
где $\widehat{u}$ — кватернион с нулевой действительной частью. %вектор единичной длинны.
Легко понять, что 
%% произведение кватернионов 
$\widehat{u}\widehat{u} = -1$. Отметим подобие: комплексное число единичной длинны может быть записано как $\cos\theta  + i\sin\theta$. Это подобие не случайно, определение Эйлера для комплексных чисел обобщается до определения кватернионов:
\begin{equation}
\exp(\widehat{u}\theta) = \cos\theta  + \widehat{u}\sin\theta ,
\end{equation}
где экспонента в левой части тождества — это символическая подстановка кватерниона $\widehat{u}\theta$ в функцию $\exp$, которая имеет смысл если $\widehat{u}\widehat{u} = -1$. С помощью этого тождества можно определить функцию возведение в степень единичного кватерниона:
\begin{equation}
q^t = (\cos\theta  + \widehat{u}\sin\theta )^t = \exp(\widehat{u}t\theta) = \cos(t\theta) + \widehat{u}\sin(t\theta) 
\end{equation}
и логарифм единичного кватерниона:
\begin{equation}
\label{eq:12}
\log(q) = \log(\cos\theta  + \widehat{u}\sin\theta ) = \log(\exp(\widehat{u} \theta)) = \widehat{u} \theta .
\end{equation}

Важно отметить, что не коммутативность операции умножения кватернионов запрещает стандартные тождества для экспоненциальной и логарифмической функций. Кватернионы $\exp(q_0)\exp(q_1)$ и $\exp(q_0 + q_1)$ необязательно равны.

\section{Связь кватернионов и поворотов \\ пространства}

Единичный кватернион $q = \cos\theta  + \widehat{u}\sin\theta $ задаёт преобразование поворота трёхмерного пространства на угол $2\theta$ вокруг оси $\widehat{u}$. На любой вектор $\widehat{v}$ преобразование определённое кватернионом $q$ действует следующим образом: $R(\widehat{v}) = q\widehat{v}q^*$. Чтобы показать, что это преобразование поворота, необходимо доказать, что $R(\widehat{v})$ трёхмерный вектор, и что функция $R$ сохраняет длину и является линейным преобразованием не содержащим отражающей компоненты.

Покажем, что $R(\widehat{v})$ трёхмерный вектор:
\begin{multline*}
W(R(\widehat{v})) = W(q\widehat{v}q^*) = [(q\widehat{v}q^*) + (q\widehat{v}q^*)^*]/2 = [q\widehat{v}q^* + q\widehat{v}^*q^*]/2 = {} \\
{} = q[(\widehat{v} + \widehat{v}^*)/2]q^* = q W(\widehat{v}) q^* = W(\widehat{v}) = 0 .
\end{multline*}

Покажем, что $R(\widehat{v})$ сохраняет длину:
\[ N(R(\widehat{v})) = N(q\widehat{v}q^*) = N(q) N(\widehat{v}) N(q^*) = N(q) N(\widehat{v}) N(q) = N(\widehat{v}) . \]

Покажем, что $R(\widehat{v})$ линейное преобразование. Пусть $\alpha$ некоторый скаляр и пусть $\widehat{v}$ и $\widehat{w}$ два вектора:
\begin{multline*}
R(\alpha \widehat{v} + \widehat{w}) = q(\alpha \widehat{v} + \widehat{w})q^* = (q \alpha \widehat{v} q^*) + (q \widehat{w} q^*) = {} \\ 
{} = \alpha (q \widehat{v} q^*) + (q \widehat{w} q^*) = \alpha R(\widehat{v}) + R(\widehat{w}) , 
\end{multline*}
то есть показано, что отображение линейной комбинации векторов эквивалентно линейной комбинации отображений векторов.

%Мы только что показали,
Получаем, что $R(\widehat{v})$ является ортогональным преобразованием. Ортогональные преобразования включают в себя  преобразования поворота и отражения.
% Зафиксируем вектор $\widehat{v}$ и 
Рассмотрим $R$ как функцию от $q$ (на множестве единичных кватернионов) : $R(q) = q\widehat{v}q^*$. Эта функция непрерывна (на $q$), и при любом (единичном) $q$ является ортогональным преобразованием. Следовательно, $\lim_{q \rightarrow 1}{R(q)} = R(1) = I$, где $I$ тождественное отображение (предел берётся вдоль произвольной траектории приближающейся к $1$ в множестве единичных кватернионов). Определитель ортогонального преобразования $R(q)$ так же является непрерывной функцией, и значит $\lim_{q\rightarrow 1}{\det(R(q))} = \det(I) = 1$. Определитель ортогонального преобразования может быть либо $1$, либо $-1$; из непрерывности функции $\det(R(q))$ следует, что определитель тождественно равен $1$, значит $R(q)$ не содержит отражающей компоненты.

Теперь покажем, что осью вращения является вектор $\widehat{u}$ и поворот происходит на угол $2\theta$. Чтобы показать первое, достаточно доказать, что $\widehat{u}$ инвариант преобразования. Вспомним что $\widehat{u}\widehat{u} = {\widehat{u}}^2 = -1$. Из этого следует $\widehat{u}^3 = - \widehat{u}$ и 
\begin{multline*}
R(\widehat{u}) = q \widehat{u} q^* = (\cos \theta + \widehat{u} \sin \theta) \widehat{u} (\cos \theta - \widehat{u} \sin \theta) = {} \\
{} = (\cos \theta)^2 \widehat{u} - (\sin \theta)^2 \widehat{u}^3 =  (\cos \theta)^2\widehat{u} - (\sin \theta)^2 (-\widehat{u}) = \widehat{u} .
\end{multline*}

Покажем, что поворот происходит на угол  $2 \theta$. Пусть $\widehat{u}$, $\widehat{v}$ и $\widehat{w}$ правый ортонормированный базис. Для этих векторов верно следующее: они все имеют единичную длину; $\widehat{u}\bullet\widehat{v} = \widehat{u}\bullet\widehat{w} = \widehat{v}\bullet\widehat{w} = 0$; и $\widehat{u}\times\widehat{v} = \widehat{w}$, $\widehat{v}\times\widehat{w} = \widehat{u}$, $\widehat{w}\times\widehat{u} = \widehat{v}$. Будем считать, что после преобразования вектор $\widehat{v}$ переходит в вектор $q\widehat{v}q^*$, повернувшись на угол $\phi$.

Используя следующие факты, что для единичных кватернионов с нулевой действительно частью $\widehat{v}^* = -\widehat{v}$ и $\widehat{v}^2 = -1$, сделаем вспомогательное построение:
\begin{align*}
\widehat{v} q \widehat{v} q^* & = \widehat{v} (\cos\theta  + \widehat{u} \sin\theta) \widehat{v} (\cos\theta - \widehat{u} \sin\theta) = {} \\
{} & = \widehat{v}^2 \cos^2\theta - \widehat{v}^2 \widehat{u} \sin\theta \cos\theta  + \widehat{v} \widehat{u} \widehat{v} \sin\theta \cos\theta  - (\widehat{v} \widehat{u})^2 \sin^2\theta = {} \\
{} & = -\cos^2\theta + \sin^2\theta + (\widehat{u} + \widehat{v} \widehat{u} \widehat{v}) \sin\theta \cos\theta.
\end{align*}
Далее, воспользуемся уравнением \ref{eq:7}: $\widehat{v} \widehat{u} = - \widehat{v} \bullet \widehat{u} + \widehat{v} \times \widehat{u} = - \widehat{w}$ и, в свою очередь, $\widehat{v} \widehat{u} \widehat{v} = - \widehat{w} \widehat{v} = \widehat{u}$. В результате получаем:
\begin{align*}
\widehat{v} q \widehat{v} q^* = - \cos^2\theta + \sin^2\theta + 2 \widehat{u} \sin\theta \cos\theta = - \cos(2 \theta) + \widehat{u} \sin(2 \theta). 
\end{align*}

Теперь покажем равенство углов $\phi$ и $2 \theta$:
\begin{multline*}
\cos \phi = \widehat{v} \bullet (q \widehat{v} q^*) = [\text{см. уравнение \ref{eq:8}}] = {} \\
{} = W[\widehat{v}^* q \widehat{v} q^*] = - W[\widehat{v} q \widehat{v} q^*] = - W[- \cos(2 \theta) + \widehat{u} \sin(2 \theta)] = \cos(2 \theta) , 
\end{multline*}
\begin{multline*}
\widehat{u} \sin \phi = \widehat{v} \times (q \widehat{v} q^*) = [\text{см. уравнение \ref{eq:7}}] = {} \\
{} = \widehat{v} q \widehat{v} q^* + \widehat{v} \bullet (q \widehat{v} q^*) = - \cos(2 \theta) + \widehat{u} \sin(2 \theta) + \cos(2 \theta) = \widehat{u} \sin(2 \theta) .
\end{multline*}
Следовательно углы $\phi$ и $2 \theta$ равны (с точностью до $2 \pi$).

Важно отметить, что кватернионы $q$ и $-q$ задают одно и тот же преобразование. Действительно, $(-q)\widehat{v}(-q)^* = q\widehat{v}q^*$. Несмотря на это, далее мы продемонстрируем, что при интерполяции имеет значение какой из кватернионов выбрать.

\section{Дифференциальное исчисление \\ кватернионов}

В теме интерполяции кватернионов нам понадобятся производные некоторых простых функций: производная произведения ($q_0 q_1$) и производная функции возведения в действительную степень единичного кватерниона ($q^t$).

Производная произведения имеет вид:
\begin{equation}
\label{eq:13}
\frac{d}{dt}(q_0(t) q_1(t)) = q_0'(t) q_1(t) + q_0(t) q_1'(t).
\end{equation}

Производная функции возведения в действительную степень единичного кватерниона ($q^t$), в случае когда кватернион $q$ фиксирован, имеет вид:
\begin{equation}
\label{eq:14}
\frac{d}{dt}q^t = q^t\log(q) ,
\end{equation}
функция $\log$ ранее уже была до-определена на множество единичных кватернионов (см. уравнение \ref{eq:12}). В случае когда степень является функцией, производная имеет вид:
\begin{equation}
\label{eq:15}
\frac{d}{dt}q^{f(t)} = f'(t)q^{f(t)}log(q) .
\end{equation}
В общем случае, когда $q$ зависит от $t$, производная имеет вид:
\begin{equation}
\label{eq:16}
\frac{d}{dt}(q(t))^{f(t)} = f'(t)(q(t))^{f(t)}log(q(t)) + f(t)(q(t))^{f(t) - 1}q'(t).
\end{equation}

Заметим, формулы производных имеют тот же вид, что и в стандартном курсе дифференциального исчисления; но для кватернионов нужно учитывать, что отсутствует коммутативность по умножению.

\section{Сферическая линейная интерполяция}

Пусть $q_0$ и $q_1$ два кватерниона единичной длинны, рассмотрим их как векторы в четырёхмерном пространстве; пусть $\theta$ угол между этими векторами.

Потребуем, чтобы при равномерном изменении $t$ между $0$ и $1$, величина $q(t)$ равномерно менялась вдоль дуги окружности от $q_0$ до $q_1$.

Определим интерполирующую функцию в виде:
\[q(t) = c_0(t) q_0 + c_1 (t)q_1, \qquad \text{для $t \in [0, 1]$} ,\]
где $c_0(t)$ и $c_1(t)$ действительные функции, $c_0(0) = 1$ и $c_1(0) = 0$, $c_0(1) = 0$, $c_1(1) = 1$. Остаётся определить функции $c_0(t)$ и $c_1(t)$, приведём простые рассуждения для их поиска.

Если $q(t)$ равномерно изменяется вдоль дуги окружности от $q_0$ до $q_1$, то угол между $q_0$ и $q(t)$ равен $t \theta$, угол между $q(t)$ и $q_1$ равен $(1 - t) \theta$.
Умножим уравнение для $q(t)$ на $q_0$, получим:
\[q(t) \bullet q_0 = \cos(t \theta) = c_0(t) + \cos(\theta) c_1(t) ,\]
так же умножим на $q_1$ :
\[q(t) \bullet q_1 = \cos((1 - t) \theta) = \cos(\theta) c_0(t) + c_1(t) .\]
Получили два линейных уравнения с двумя неизвестными $c_0$ и $c_1$. Решением системы относительно $c_0$ будет:
\[c_0(t) = \frac{\cos(t \theta) - \cos\theta \cos((1 - t) \theta)}{1 - \cos^2\theta} = \frac{\sin((1 - t) \theta)}{\sin\theta} .\]
Последнее равенство получается применением формул тригонометрии. Формулу для $c_1$ можно получить из симметрии: $c_1(t) = c_0(1 - t)$, либо решая систему:
\[c_1(t) = \frac{\cos((1 - t) \theta) - \cos\theta \cos(t \theta)}{1 - \cos^2\theta} = \frac{\sin(t \theta)}{\sin\theta} .\]

Функция сферической линейной интерполяции обозначается $\slerp$ и определяется
как:
\begin{equation}
\label{eq:17}
\slerp(t; q_0, q_1) = \frac{q_0 \sin((1-t) \theta) + q_1 \sin(t \theta)}{\sin\theta} , \qquad \text{для $t \in [0, 1] $} .
\end{equation}

Ранее мы отмечали, что $q_1$ и $-q_1$ определяют одно и тот же преобразование поворота; но величины $\slerp(t; q_0, q_1)$ и $\slerp(t; q_0, -q_1)$ не эквивалентны. Знак $\sigma$ для $q_1$ обычно выбирают так, чтобы $q_0 \bullet (\sigma q_1) \geq 0$ (чтобы угол между $q_0$ и $\sigma q_1$ был острым); это позволяет избежать при интерполяции дополнительных вращений.

Для единичных кватернионов $\slerp$ может быть определён как:
\begin{equation}
\label{eq:18}
\slerp(t; q_0, q_1) = q_0 (q_0^{-1} q_1)^t .
\end{equation}
Основная идея такой записи состоит в том, что: $q_1 = q_0 (q_0^{-1} q_1)$; $q_0^{-1} q_1 = \cos\theta + \widehat{u} \sin\theta$, где $\theta$ угол между $q_0$ и $q_1$. Параметр %времени
$t$ должен быть введён так, чтобы равномерно (вдоль дуги окружности) корректировать $q_0$ между $q_0$ и $q_1$. Таким образом, получаем нечто вроде: $q(t) = q_0 [\cos(t \theta) + \widehat{u} \sin(t \theta)] = q_0 [\cos\theta + \widehat{u} \sin\theta]^t = q_0(q_0^{-1}q_1)^t$.

Производную для $\slerp$ представленной в виде \ref{eq:18} можно получить используя уравнение \ref{eq:14}:
\begin{equation}
\label{eq:19}
\slerp'(t; q_0, q_1) = q_0 (q_0^{-1} q_1)^t \log(q_0^{-1}q_1) .
\end{equation}

\section{Сферическая кубическая интерполяция}

Сферическая кубическая интерполяция кватернионов получается с помощью приёма описанного в \cite{Boehm}. Данный метод имеет оттенок билинейной интерполяции на четырёхугольнике и состоит в последовательном применении  трёх линейных интерполяций (отметим сходство с алгоритмом de Casteljau \cite{GerlandFarin}).

Пусть имеются четыре кватерниона $p$, $a$, $b$ и $q$, представим их как упорядоченную четвёрку вершин четырёхугольника. Основная идея алгоритма состоит в следующем: пусть точка $c$ интерполирует {\it ребро} из $p$ в $q$% используя $\slerp$
, а точка $d$ интерполирует {\it ребро} из $a$ в $b$; теперь интерполируем кромку интерполяции $c$ и $d$ чтобы получить окончательный результат $e$ .

Функция сферической кубической интерполяция обозначается $\squad$ и определяется как:
\begin{multline}
\label{eq:20}
\squad(t; p, a, b, q) = {} \\
{} = \slerp[2 t (1 - t); \slerp(t; p, q), \slerp(t; a, b)] , \qquad \text{для $t \in [0, 1]$ .}
\end{multline}

Для единичных кватернионов, используя уравнение \ref{eq:18}, можно получить 
% аналогичную
другую формулу для $\squad$:
\begin{equation}
\label{eq:21}
\squad(t; p, a, b, q) = \slerp(t; p, q) [(\slerp(t; p, q))^{-1} \slerp(t; a, b)]^{2 t (1 - t)} .
\end{equation}

Производная для функции $\squad$, представленной в виде \ref{eq:21}, может быть получена с применением уравнений \ref{eq:13} — \ref{eq:16}. Для того, чтобы упростить запись, введём обозначения:
\[ U(t) = \slerp(t; p, q) \qquad V(t) = \slerp(t; a, b) \qquad W(t) = (U(t))^{-1} V(t) .\]
Перепишем функцию $\squad$:
\[ \squad(t; p, a, b, q) = \slerp(2 t (1 - t); U, V) = U W^{2 t (1 - t)} . \]
Теперь можно выписать производную:
\begin{multline}
\label{eq:22}
\squad'(t; p, a, b, q) = \frac{d}{dt}[UW^{2 t (1 - t)}] = U \frac{d}{dt}[W^{2 t (1 - t)}] + U' [W^{2 t (1 - t)}] = {} \\
{} = U[(2 - 4 t) W^{2 t (1 - t)} \log{}W + 2 t (t - 1) W^{2 t (t - 1) - 1} W'] + U' [W^{2 t (1 - t)}] 
\end{multline}

В сплайн интерполяции 
%будет использоваться $\squad$, и 
нам потребуются значения производной функции $\squad$ при $t = 0$ и $t = 1$. Рассмотрим значения функций $U$ и $V$, $W$, $U' = U \log(p^{-1} q)$ на концах отрезка [0, 1]:
\[ U(0) = p \qquad V(0) = a \qquad W(0) = p^{-1} a \qquad U'(0) = p \log(p^{-1} q) \]
\[ U(1) = q \qquad V(1) = b \qquad W(1) = q^{-1} b \qquad U'(1) = q \log(p^{-1} q) .\]
Этого достаточно для того, чтобы посчитать значения производной функции $\squad$ на концах отрезка [0, 1]:
\begin{equation}
\label{eq:23}
\begin{split}
\squad'(0; p, a, b, q) & = p[\log(p^{-1}q) + 2\log(p^{-1}a)] \\
\squad'(1; p, a, b, q) & = q[\log(p^{-1}q) - 2\log(q^{-1}b)] .
\end{split}
\end{equation}

\section{Сплайн интерполяция кватернионов}

Пусть имеется последовательность из $N$ единичных кватернионов $\{q_n\}^{N - 1}_{n = 0}.$ Необходимо построить сплайн, проходящий через все ключевые точки, и чтобы его производная была непрерывна. Основная идея состоит в следующем: сегменты сплайна строятся с помощью сферической кубической интерполяции; выбором промежуточных кватернионов $a_n$ и $b_n$, обеспечивается непрерывность производной. 

Каждый сегмент сплайна устроен следующим образом:
%\[
\begin{equation}
\label{eq:24}
S_n(t) = \squad(t; q_n, a_n, b_{n + 1}, q_{n + 1}).
\end{equation}
%\]
Из свойств функции $\squad$ следует непрерывность сплайна: 
\[ S_{n - 1}(1) = q_n = S_n(0). \]
Чтобы производная сплайна была непрерывной нужно равенство производных последовательных сегментов сплайна в точке перехода:
\[ S_{n - 1}'(1) = S_n'(0) .\]

Используя уравнения \ref{eq:23} найдём значения производных сегментов сплайна в точке перехода: 
\[ S_{n - 1}'(1) = q_n [\log(q_{n - 1}^{-1} q_n) - 2 \log(q_n^{-1} b_n)], \]
\[ S_n'(0) = q_n [\log(q_n^{-1} q_{n + 1}) + 2 \log(q_n^{-1} a_n)] .\]
Условие непрерывности производной содержит одно уравнение и два неизвестных $a_n$ и $b_n$, следовательно мы имеем одну степень свободы. В работе \cite{KenShoemake} даётся совет, выбирать значение производной в точке перехода
%равным среднему значению касательных $S_{n - 1}'(1) = q_n T_n = S_n'(0)$, где:
равным $S_{n - 1}'(1) = S_n'(0) = q_n T_n$, где:
%\begin{equation}
\[ T_n = \frac{\log(q_n^{-1} q_{n + 1}) + \log(q_{n-1}^{-1} q_n)}{2} . \]
%\end{equation}

Мы имеем два уравнения чтобы определить коэффициенты $a_n$ и $b_n$; несложно
показать, что:
\begin{equation}
\label{eq:25}
a_n = b_n = q_n \exp(-\frac{\log(q_n^{-1}q_{n + 1}) - \log(q_{n-1}^{-1}q_n)}{4}) .
\end{equation}
Таким образом: $S_n(t) = \squad(t; q_n, a_n, a_{n + 1}, q_{n+1})$.

Пример. Чтобы показать кубическую природу интерполяции, проведём рассуждения для последовательности кватернионов заданной выражением: $q_n = \exp(i \theta_n)$ . Это последовательность комплексных чисел, следовательно умножение коммутативно и можно пользоваться обычными свойствами экспоненты и логарифма.

Промежуточные точки имеют следующие значения: 
\[a_n = \exp(-\frac{i}{4}(\theta_{n + 1} - 6\theta_n + \theta_{n - 1})) .\]

Найдём функцию задающую сегмент сплайна:
\[ \slerp(t; q_n, q_{n + 1}) = \exp(i((1 - t)\theta_n + t \theta_{n + 1})) , \]
\begin{multline*}
\slerp(t; a_n, a_{n + 1}) = \exp(-\frac{i}{4} ((1 - t)(\theta_{n + 1} - 6\theta_n + \theta_{n - 1}) + {} \\
{} + t(\theta_{n + 2} - 6\theta_{n + 1} + \theta_n)))
\end{multline*}
и окончательный результат:
\begin{multline*}
\squad(t; q_n, a_n, a_{n + 1}, q_{n + 1}) = \exp(i([1 - 2 t (1 - t)][(1 -  t) \theta_n + t \theta_{n + 1}] - {} \\
{} - \frac{2 t (1 - t)}{4}[(1 - t)(\theta_{n + 1} - 6 \theta_n + \theta_{n - 1}) + t (\theta_{n + 2} - 6 \theta_{n + 1} + \theta_n)])).
\end{multline*}
Угловая функция кубической интерполяции имеет вид:
\begin{multline*}
\phi(t) = \frac{1}{2} ((t^3 - t^2) \theta_{n + 2} - (3 t^3 - 4 t^2 - t) \theta_{n + 1} + {} \\
{} + (3 t^3 - 5 t^2 + 2) \theta_n - (t^3 - 2 t^2 + t) \theta_{n - 1}) .
\end{multline*}

Легко показать, что $\phi(0) = \theta_n$ и $\phi(1) = \theta_{n+1}$, $\phi'(0) = (\theta_{n + 1} - \theta_{n - 1}) / 2$, $\phi'(1) = (\theta_{n + 2} - \theta_n) / 2$. Производные на концах равны центрированной разнице, среднему левой и правой производной, что соответствует теории.

\begin{thebibliography}{99}
\bibitem{DavidEberly}
David Eberly, Quaternion Algebra and Calculus, 
http://www.geometrictools.com
\bibitem{KenShoemake}
Ken Shoemake, Animating rotation with quaternion calculus, ACM
SIGGRAPH 1987, Course Notes 10, Computer Animation: 3-D Motion,
Specification, and Control.
\bibitem{Boehm}
W. Boehm, On cubics: a survey, Computer Graphics and Image
Processing, vol. 19, pp. 201-226, 1982
\bibitem{GerlandFarin}
Gerald Farin, Curves and Surfaces for Computer Aided Geometric
Design, Academic Press, Inc., San Diego CA, 1990 
\end{thebibliography}

\end{document}
