% 2009-01-11

\documentclass[draft]{article}
%%
% Пакет позволяющий определить, что используется: latex или pdflatex?
%%
\usepackage{ifpdf}

%%
% Пакеты AMS*
%%
\usepackage{amsfonts, amsmath, amsthm}

%%
% Определяем, используется ли XeTex?
%%
\ifx\XeTeXversion\undefined
  %%
  % Набор пакетов для работы с графическими файлами
  %%
  \ifpdf
    \usepackage[pdftex]{graphicx}
    \usepackage{cmap}
  \else
    \usepackage{graphicx}
  \fi

  \usepackage[utf8x]{inputenc} 

\else
  %%
  % Набор пакетов для поддержки русского языка в XeLaTeX
  %%
  \usepackage[cm-default]{fontspec}
  \usepackage{xunicode}
  \usepackage{xecyr}

  % Setting default fonts
  \usepackage{unicode-math}
  \setmainfont[Mapping=tex-text]{Cambria}
  \setmathfont{Cambria Math}
  
  %\setmainfont{Times New Roman} 
  %\setmainfont{Georgia}
  %\setmainfont{Lucida Console}
\fi

%%
% Дополнительные настройки
%%
\usepackage[english,russian]{babel} 
\usepackage{indentfirst}

%%
% Вспомогательные команды
%%
\newcommand{\vect}[2]{\left[\begin{array}{cc}#1 & #2\\ \end{array}\right]}
\newcommand{\matr}[4]{\left[\begin{array}{cc}#1 & #2\\ #3 & #4 \\ \end{array}\right]}

%%
% Тело документа
%%
\begin{document}

\section{Задача}

Имеется n корзин. Независимо друг от друга в корзины бросают $k$ мячей. Каждый мяч попадает в любую из корзин с вероятностью $\frac{1}{n}$. Найти математическое ожидание числа непустых корзин.

\section{Решение}

Введём обозначения: $M_k$ --- математическое ожидание числа непустых корзин при $k$ брошенных мячах, $p_{{k}{i}}$ --- вероятность того, что после $k$ бросков ровно $i$ корзин не пусты.

Сделаем несколько простых замечаний про величину $p_{{k}{i}}$:
\begin{itemize}
\item[\it(i)]   $p_{{0}{0}} = 1;$
\item[\it(ii)]  $p_{{k}{0}} = 0,$ если $k \not= 0;$
\item[\it(iii)] $p_{{k}{i}} = 0,$ если $i > k.$
\end{itemize}

Теперь мы можем указать индуктивную формулу
\[
  p_{{k+1}{i}} = p_{{k}{i-1}}\frac{n-(i-1)}{n} + p_{{k}{i}}\frac{i}{n}.
\]

Воспользуемся формулой для вычисления математического ожидания:
\begin{eqnarray*}
  M_{k+1} & = & \sum_{i=1}^{k+1}{i p_{{k+1}{i}}} = \sum_{i=1}^{k+1}{i (p_{{k}{i-1}}\frac{n-(i-1)}{n} + p_{{k}{i}}\frac{i}{n})} = \\
          & = & \sum_{i=1}^{k+1}{i (p_{{k}{i-1}}\frac{n}{n} - p_{{k}{i-1}}\frac{i-1}{n} + p_{{k}{i}}\frac{i+1}{n} - p_{{k}{i}}\frac{1}{n})} = \\
          & = & \sum_{i=1}^{k+1}{i p_{{k}{i-1}} - \frac{1}{n}\sum_{i=1}^{k+1}{i p_{{k}{i}}}} - \sum_{i=1}^{k+1}{\frac{(i-1)i}{n} p_{{k}{i - 1}}} + \sum_{i=1}^{k+1}{\frac{i(i+1)}{n} p_{{k}{i}}}.
\end{eqnarray*}

Для упрощения дальнейшего изложения, преобразуем выражение по частям:
\begin{eqnarray*}
  \sum_{i=1}^{k+1}{i p_{{k}{i-1}} - \frac{1}{n}\sum_{i=1}^{k+1}{i p_{{k}{i}}}} & \stackrel{(ii,iii)}{=} & \sum_{i=2}^{k+1}{i p_{{k}{i-1}} - \frac{1}{n}\sum_{i=1}^{k}{i p_{{k}{i}}}} = \\
          & = & \sum_{i=1}^{k}{(i + 1) p_{{k}{i}} - \frac{1}{n}\sum_{i=1}^{k}{i p_{{k}{i}}}} = \\
          & = & \sum_{i=1}^{k}{i p_{{k}{i}} + \sum_{i=1}^{k}{p_{{k}{i}}} - \frac{1}{n}\sum_{i=1}^{k}{i p_{{k}{i}}}} = \\
          & = & M_k + 1 - \frac{1}{n}M_k = \frac{n - 1}{n}M_k + 1.
\end{eqnarray*}
\begin{eqnarray*}
\sum_{i=1}^{k+1}{\frac{(i-1)i}{n} p_{{k}{i - 1}}} - \sum_{i=1}^{k+1}{\frac{i(i+1)}{n} p_{{k}{i}}} & \stackrel{(iii)}{=} & \sum_{i=2}^{k+1}{\frac{(i-1)i}{n} p_{{k}{i - 1}}} - \sum_{i=1}^{k}{\frac{i(i+1)}{n} p_{{k}{i}}} = \\
          & = & \sum_{i=1}^{k}{\frac{i(i+1)}{n} p_{{k}{i}}} - \sum_{i=1}^{k}{\frac{i(i+1)}{n} p_{{k}{i}}} = \\
          & = & 0.
\end{eqnarray*}
В итоге мы получаем следующую формулу:
\[
  M_{k+1} = \frac{n-1}{n}M_k + 1.
\]

Перепишем формулу в матричном виде:
\[
  \vect{M_{k+1}}{1} = \vect{M_{k}}{1}\matr{\frac{n-1}{n}}{0}{1}{1}.
\]
Теперь не сложно понять почему верна следующая формула:
\[
  \vect{M_k}{1} = \vect{0}{1}{\matr{\frac{n-1}{n}}{0}{1}{1}}^k,
\]
достаточно догадаться, что $M_0 = 0.$

Теперь найдём собственные числа и собственные вектора матрицы:
\[
  \lambda_1 = \frac{n-1}{n}, \lambda_2 = 1 \qquad v_1 = \vect{1}{0}, v_2 = \vect{n}{1}.
\]
Перепишем матрицу используя {\it знание} её собственных значений:
\[
  \matr{\frac{n-1}{n}}{0}{1}{1} = \matr{1}{0}{-n}{1} \matr{\frac{n-1}{n}}{0}{0}{1} \matr{1}{0}{n}{1},
\]
при этом
\[
  \matr{1}{0}{-n}{1} \matr{1}{0}{n}{1} = \matr{1}{0}{0}{1}.
\]

Теперь можно вернуться к нашей матричной формуле и улучшить её:
\begin{eqnarray*}
  \vect{M_k}{1} & = & \vect{0}{1}{\matr{\frac{n-1}{n}}{0}{1}{1}}^k = \\
          & = & \vect{0}{1}\left(\matr{1}{0}{-n}{1} \matr{\frac{n-1}{n}}{0}{0}{1} \matr{1}{0}{n}{1}\right)^k = \\
          & = & \vect{0}{1}\matr{1}{0}{-n}{1} {\matr{\frac{n-1}{n}}{0}{0}{1}}^k \matr{1}{0}{n}{1} = \\
          & = & \vect{0}{1}\matr{1}{0}{-n}{1} \matr{(\frac{n-1}{n})^k}{0}{0}{1} \matr{1}{0}{n}{1} = \\
          & = & \vect{-n}{1} \matr{(\frac{n-1}{n})^k}{0}{0}{1} \matr{1}{0}{n}{1} = \\
          & = & \vect{-n(\frac{n-1}{n})^k}{1} \matr{1}{0}{n}{1} = \\
          & = & \vect{n(1-(\frac{n-1}{n})^k)}{1}. \\
\end{eqnarray*}

Окончательная формула для вычисления математического ожидания:
\[
  \boxed{M_k = n(1 - \left(\frac{n-1}{n}\right)^k)}
\]

\end{document}

