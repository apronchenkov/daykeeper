% 2009-01-14
% 2009-01-11
% 2009-01-10

\documentclass[a4paper,10pt]{article}

%%
% Пакет позволяющий определить, что используется: latex или pdflatex?
%%
\usepackage{ifpdf}

%%
% Пакеты AMS*
%%
\usepackage{amsfonts, amsmath, amsthm}

%%
% Определяем, используется ли XeTex?
%%
\ifx\XeTeXversion\undefined
  %%
  % Набор пакетов для работы с графическими файлами
  %%
  \ifpdf
    \usepackage[pdftex]{graphicx}
    \usepackage{cmap}
  \else
    \usepackage{graphicx}
  \fi

  \usepackage[utf8x]{inputenc} 

\else
  %%
  % Набор пакетов для поддержки русского языка в XeLaTeX
  %%
  \usepackage[cm-default]{fontspec}
  \usepackage{xunicode}
  \usepackage{xecyr}

  % Setting default fonts
  \usepackage{unicode-math}
  \setmainfont[Mapping=tex-text]{Cambria}
  \setmathfont{Cambria Math}
  
  %\setmainfont{Times New Roman} 
  %\setmainfont{Georgia}
  %\setmainfont{Lucida Console}
\fi

%%
% Дополнительные настройки
%%
\usepackage[english,russian]{babel} 
\usepackage{indentfirst}


\newcommand{\exercize}[1]{\textbf{#1.}}
\newcommand{\abs}[1]{\lvert{}#1{}\rvert}

%%
% Начало документа
%%
\begin{document}


В задачах 792 -- 812 найти производные функций $y$, заданных не явно.

\exercize{794} \[ x^3 + y^3 - 3 a x y = 0. \]

\exercize{795} \[ y^2 \cos{x} = a^2 \sin{3x}. \]

\exercize{802} \[ 2 y \ln{y} = x. \]

\exercize{803} \[ x - y = \arcsin{x} - \arcsin{y}. \]

\exercize{810} \[ \tg{\frac{y}{2}} = \sqrt{\frac{1 - k}{1 + k}} \tg{\frac{x}{2}}. \]

\exercize{843} Показать, что касательные проведённые к гиперболе
\[ y = \frac{x - 4}{x - 2} \]
в точках её пересечения с осями координат, параллельны между собой.

\exercize{933} Построить графики фукций, заданых параметрически:
\begin{itemize}
\item[а)] $x = 3 \cos{t},\ y = 4 \sin{t};$
\item[б)] $x = t^2 - 2 t,\ y = t^2 + 2 t;$
\item[в)] $x = \cos{t},\ y = t + 2 \sin{t};$
\item[г)] $x = 2^{t - 1},\ y = \frac{1}{4}(t^3 + 1).$
\end{itemize}

\exercize{934} Из уравеий, параметрически задающих функцию, исключить параметр:
\begin{itemize}
\item[1)] $x = 3 t,\ y = 6 t - t^2;$
\item[2)] $x = \cos{t},\ y = \sin{2 t}.$
\end{itemize}

\exercize{935} Найти значение параметра, соответствующее заданым координатам точки а лиии, уравнение которой дано в параметрической форме:
\begin{itemize}
\item[1)] $x = 3 (2 \cos{t} - \cos{2 t}),\ y = 3 (2 \sin{t} - \sin{2t});\ (-9, 0);$
\item[2)] $x = t^2 + 2t,\ y = t^3 + t;\ (3, 2).$
\end{itemize}

В задачах 936 -- 945 найти производные от $y$ по $x.$

\exercize{941} \[ x = \ln(1 + t^2),\ y = t - \arctg{t}. \]

В задачах 946 -- 949 найти угловые коэффициенты касательных к данным линиям.

\nopagebreak

\exercize{946} \[ x = 3 \cos{t},\ y = 4 \sin{t} \ \textrm{в точке}\ (3 \sqrt{2}/2, 2 \sqrt{2}). \]

\exercize{948} \[ x = t^3 + 1,\ y = t^2 + t + 1 \ \textrm{в точки}\ (1, 1). \]

\exercize{952} Убедиться в том, что функция, заданная параметрически уравнениями
\[ x = \frac{1 + t}{t^3},\ y = \frac{3}{2 t^2} + \frac{2}{t}, \]
удовлетворяет соотношению
\[ x y' = 1 + y' \ \left( y' = \frac{dy}{dx} \right). \]

В задачах 963 -- 966 составить уравнеия касательной и нормали к даным линиям в указанных точках.

\exercize{963} \[ x = 2 e^{t},\ y = e^{-t}\ \textrm{при}\ t = 0. \]

\exercize{965} \[ x = 2 \ln{\ctg{t}} + 1,\ y = \tg{t} + \ctg{t}\ \textrm{при}\ t = \pi/4. \]

\exercize{1010} \[ y = (x^2 + 1)^3;\ y'' = ? \]

\exercize{1013} \[ f(x) = \arctg{x}; f''(1) = ? \]

В задачах 1019 -- 1028 найти вторые производные от фукций.

\exercize{1023} \[ y = \ln(x + \sqrt{1 + x^2}). \]

В задачах 1029 -- 1040 найти общие выражения для производных порядка $n$ от функций:

\exercize{1029} \[ y = e^{a x}. \]

\exercize{1031} \[ y = \sin{a x} + \cos{b x}. \]

\exercize{1033} \[ y = x e^x. \]

\exercize{1042} Доказать, что функция 
\[ y = \frac{x - 3}{x + 4} \]
удовлетворяет соотошению
\[ y'' + 2 y' + 2y = 0. \]

\exercize{1056} \[ b^2 x^2 + a^2 y^2 = a^2 b^2;\ \frac{d^2 y}{{dx}^2} = ? \]

\exercize{1058} \[ y = \tg{(x + y)};\ \frac{d^3 y}{{dx}^3} = ? \]

\exercize{1069} \[ x = a t^2,\ y = b t^3;\ \frac{d^2 x}{{dy}^2} = ? \]

\exercize{1070} \[ x = a \cos{t},\ y = a \sin{t};\ \frac{d^2 x}{{dy}^2} = ? \]

\exercize{1100} \[ y = \arctg{\left(\frac{b}{a} \tg{x}\right)};\ d^2 y = ? \]

\exercize{1109, 1176, 1179, 1197, 1199, 1200, 1203, 1325, 1334, 1344, 1358, 1363, 1398, 1402, 1412, 1416, 1417, 1425}

\end{document}
