% 2009-01-14
% 2009-01-11
% 2009-01-10

\documentclass[a4paper,10pt]{article}

%%
% Пакет позволяющий определить, что используется: latex или pdflatex?
%%
\usepackage{ifpdf}

%%
% Пакеты AMS*
%%
\usepackage{amsfonts, amsmath, amsthm}

%%
% Определяем, используется ли XeTex?
%%
\ifx\XeTeXversion\undefined
  %%
  % Набор пакетов для работы с графическими файлами
  %%
  \ifpdf
    \usepackage[pdftex]{graphicx}
    \usepackage{cmap}
  \else
    \usepackage{graphicx}
  \fi

  \usepackage[utf8x]{inputenc} 

\else
  %%
  % Набор пакетов для поддержки русского языка в XeLaTeX
  %%
  \usepackage[cm-default]{fontspec}
  \usepackage{xunicode}
  \usepackage{xecyr}

  % Setting default fonts
  \usepackage{unicode-math}
  \setmainfont[Mapping=tex-text]{Cambria}
  \setmathfont{Cambria Math}
  
  %\setmainfont{Times New Roman} 
  %\setmainfont{Georgia}
  %\setmainfont{Lucida Console}
\fi

%%
% Дополнительные настройки
%%
\usepackage[english,russian]{babel} 
\usepackage{indentfirst}


\newcommand{\exercize}[1]{\textbf{#1.}}
\newcommand{\abs}[1]{\lvert{}#1{}\rvert}

%%
% Начало документа
%%
\begin{document}

\section{Домашнее задание}

\exercize{933} Построить графики фукций, заданых параметрически:
\begin{itemize}
\item[б)] $x = t^2 - 2 t,\ y = t^2 + 2 t;$
\item[в)] $x = \cos{t},\ y = t + 2 \sin{t};$
\item[г)] $x = 2^{t - 1},\ y = \frac{1}{4}(t^3 + 1).$
\end{itemize}

В задачах 946 -- 949 найти угловые коэффициенты касательных к данным линиям.

\exercize{946} \[ x = 3 \cos{t},\ y = 4 \sin{t} \ \textrm{в точке}\ (3 \sqrt{2}/2, 2 \sqrt{2}). \]

\exercize{948} \[ x = t^3 + 1,\ y = t^2 + t + 1 \ \textrm{в точки}\ (1, 1). \]

\exercize{1010} \[ y = (x^2 + 1)^3;\ y'' = ? \]

\exercize{1013} \[ f(x) = \arctg{x}; f''(1) = ? \]


\section{Работа на занятии}

В задачах 1019 -- 1028 найти вторые производные от фукций.

\exercize{1023} \[ y = \ln(x + \sqrt{1 + x^2}). \]

В задачах 1029 -- 1040 найти общие выражения для производных порядка $n$ от функций:

\exercize{1029} \[ y = e^{a x}. \]

\exercize{1031} \[ y = \sin{a x} + \cos{b x}. \]

\exercize{1033} \[ y = x e^x. \]

\exercize{1042} Доказать, что функция 
\[ y = e^x \sin{x} \]
удовлетворяет соотошению
\[ y'' - 2 y' + 2y = 0, \]
а функция 
\[ y = e^{-x} \sin{x} \]
--- соотношению
\[ y'' + 2 y' + 2y = 0. \]

\exercize{1056} \[ b^2 x^2 + a^2 y^2 = a^2 b^2;\ \frac{d^2 y}{{dx}^2} = ? \]

\exercize{1058} \[ y = \tg{(x + y)};\ \frac{d^3 y}{{dx}^3} = ? \]

\exercize{1069} \[ x = a t^2,\ y = b t^3;\ \frac{d^2 x}{{dy}^2} = ? \]

\exercize{1070} \[ x = a \cos{t},\ y = a \sin{t};\ \frac{d^2 x}{{dy}^2} = ? \]

\exercize{1100} \[ y = \arctg{\left(\frac{b}{a} \tg{x}\right)};\ d^2 y = ? \]

В задачах 1324 -- 1364 найти пределы.

\exercize{1325} \[ \lim_{x \rightarrow 0} {\frac{\ln{\cos{x}}}{x}}. \]

\exercize{1334} \[ \lim_{x \rightarrow 0} {\frac{e^{x^2} - 1}{\cos{x} - 1}}. \]

\exercize{1344} \[ \lim_{x \rightarrow 0} {\frac{\ln{x}}{\ln{\sin{X}}}}. \]

\exercize{1358} \[ \lim_{x \rightarrow 0} {x^{\sin{x}}}. \]

\exercize{1363} \[ \lim_{x \rightarrow 0} {\left( 1 + \frac{1}{x^2} \right)^x}. \]

\exercize{1109, 1176, 1179, 1197, 1199, 1200, 1203, 1398, 1402, 1412, 1416, 1417, 1425}

\end{document}
