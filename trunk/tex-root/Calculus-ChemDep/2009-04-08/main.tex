% 2009-01-14
% 2009-01-11
% 2009-01-10

\documentclass[a4paper,10pt]{article}

%%
% Пакет позволяющий определить, что используется: latex или pdflatex?
%%
\usepackage{ifpdf}

%%
% Пакеты AMS*
%%
\usepackage{amsfonts, amsmath, amsthm}

%%
% Определяем, используется ли XeTex?
%%
\ifx\XeTeXversion\undefined
  %%
  % Набор пакетов для работы с графическими файлами
  %%
  \ifpdf
    \usepackage[pdftex]{graphicx}
    \usepackage{cmap}
  \else
    \usepackage{graphicx}
  \fi

  \usepackage[utf8x]{inputenc} 

\else
  %%
  % Набор пакетов для поддержки русского языка в XeLaTeX
  %%
  \usepackage[cm-default]{fontspec}
  \usepackage{xunicode}
  \usepackage{xecyr}

  % Setting default fonts
  \usepackage{unicode-math}
  \setmainfont[Mapping=tex-text]{Cambria}
  \setmathfont{Cambria Math}
  
  %\setmainfont{Times New Roman} 
  %\setmainfont{Georgia}
  %\setmainfont{Lucida Console}
\fi

%%
% Дополнительные настройки
%%
\usepackage[english,russian]{babel} 
\usepackage{indentfirst}


\newcommand{\exercize}[1]{\textbf{#1.}}
\newcommand{\abs}[1]{\lvert{}#1{}\rvert}

%%
% Начало документа
%%
\begin{document}

\section{Работа на занятии}

В задачах 1676-1702, воспользовавшись основной таблицей интегралов и простейшими правилами интегрирования, найти интегралы.

\exercize{1676} \[ \int{\sqrt{x}dx.} \]

\exercize{1678} \[ \int{\frac{dx}{x^2}.} \]

\exercize{1679} \[ \int{10^x dx.} \]

\exercize{1680} \[ \int{a^x e^x dx.} \]

\exercize{1683} \[ \int{3.4 x^{-0.17} dx.} \]

\exercize{1684} \[ \int{(1 - 2 u) du.} \]

\exercize{1688} \[ \int{\left(\frac{1 - z}{z}\right)^2 dz.} \]

\exercize{1696} \[ \int{ \tg^2{x} dx.} \]

\exercize{1702} \[ \int{ (\arcsin{x} + arccos{x}) dx.} \]

В задачах 1703-1780 найти интегралы, воспользовавшись теоремой об инвариантности формул интегрирования.

\exercize{1706} \[ \int{(x + 1)^{15} dx.} \]

\exercize{1707} \[ \int{\frac{dx}{(2x - 3)^5}.} \]

\exercize{1710} \[ \int{\sqrt{8 - 2 x} dx.} \]

\exercize{1713} \[ \int{x \sqrt{1 - x^2} dx.} \]

\exercize{1714} \[ \int{x^2 \sqrt[5]{x^3 + 2} dx.} \]

\exercize{1715} \[ \int{\frac{x dx}{\sqrt{x^2 + 1}}.} \]

\exercize{1719} \[ \int{\sin^3{x} \cos{x} dx.} \]

\exercize{1720} \[ \int{\frac{\sin{x} dx}{\cos^2{x}}.} \]

\exercize{1722} \[ \int{\cos^3{x} \sin{2 x} dx.} \]

\exercize{1729} \[ \int{\cos{3x} dx.} \]

\exercize{1734} \[ \int{e^x \sin{(e^x)} dx.} \]

\exercize{1737} \[ \int{\frac{(2x - 3) dx}{x^2 - 3 x + 8}.} \]

\exercize{1744} \[ \int{\tg{x} dx.} \]

\exercize{1749} \[ \int{\frac{dx}{x \ln{x}}.} \]

\exercize{1752} \[ \int{e^{\sin{x}} \cos{x} dx.} \]

\exercize{1756} \[ \int{e^{x^2} x dx.} \]

\exercize{1760} \[ \int{\frac{dx}{1 + 9 x^2}.} \]


В задачах 1781-1790 найти интегралы, выделив целую часть подынтегральной дроби.

\exercize{1783} \[ \int{\frac{A x}{a + b x} dx.} \]


\section{Домашнее задание}

В задачах 1676-1702, воспользовавшись основной таблицей интегралов и простейшими правилами интегрирования, найти интегралы.

\exercize{1677} \[ \int{\sqrt{m}{x^n}dx.} \]

\exercize{1685} \[ \int{(\sqrt{x} + 1)(x - \sqrt{x} + 1) dx.} \]

В задачах 1703-1780 найти интегралы, воспользовавшись теоремой об инвариантности формул интегрирования.

\exercize{1708} \[ \int{\frac{dx}{(a + bx)^c}.} \]

\exercize{1712} \[ \int{2 x \sqrt{x^2 + 1} dx.} \]

\exercize{1716} \[ \int{\frac{x^4 dx}{\sqrt{4 + x^5}}.} \]

\exercize{1721} \[ \int{\frac{\cos{x} dx}{\sqrt[3]{\sin^2{x}}}.} \]

\exercize{1723} \[ \int{\frac{\sqrt{\ln{x}}}{x} dx.} \]

\exercize{1731} \[ \int{\sin{(2x - 3)}dx.} \]

\exercize{1760} \[ \int{\frac{dx}{1 + 9 x^2}.} \]


\end{document}
