% 2009-01-11

\documentclass[draft]{article}
%%
% Пакет позволяющий определить, что используется: latex или pdflatex?
%%
\usepackage{ifpdf}

%%
% Пакеты AMS*
%%
\usepackage{amsfonts, amsmath, amsthm}

%%
% Определяем, используется ли XeTex?
%%
\ifx\XeTeXversion\undefined
  %%
  % Набор пакетов для работы с графическими файлами
  %%
  \ifpdf
    \usepackage[pdftex]{graphicx}
    \usepackage{cmap}
  \else
    \usepackage{graphicx}
  \fi

  \usepackage[utf8x]{inputenc} 

\else
  %%
  % Набор пакетов для поддержки русского языка в XeLaTeX
  %%
  \usepackage[cm-default]{fontspec}
  \usepackage{xunicode}
  \usepackage{xecyr}

  % Setting default fonts
  \usepackage{unicode-math}
  \setmainfont[Mapping=tex-text]{Cambria}
  \setmathfont{Cambria Math}
  
  %\setmainfont{Times New Roman} 
  %\setmainfont{Georgia}
  %\setmainfont{Lucida Console}
\fi

%%
% Дополнительные настройки
%%
\usepackage[english,russian]{babel} 
\usepackage{indentfirst}

%%
% Тело документа
%%
\begin{document}


\section{Линейная рекуррентная последовательность}

Пусть даны первые члены рекуррентной последовательности
\[ a_{0}, a_{-1}, a_{-2}, \ldots, a_{1-n} \]
и линейные коэффициенты
\[ \alpha_1, \alpha_2, \alpha_3, \ldots, \alpha_n. \]

Формула общего члена последовательности имеет вид
\[ a_k = \alpha_1 a_{k-1} + \alpha_2 a_{k-2} + \ldots + \alpha_n a_{k - n} = \sum_{i=1}^{n}{\alpha_i a_{k-i}}. \]


\section{Геометрическая прогрессия}

Определим соответствующую геометрическую прогрессию, для этого укажем первый член прогессии
\[ \vec{a}_{0} = (a_{0}, a_{-1}, a_{-2}, \ldots, a_{1-n}) \]
и знаменатель прогрессии
\begin{displaymath}
  Q = \left[
  \begin{array}{cccccc}
    \alpha_1 & 1      & 0      & 0      & \ldots & 0      \\
    \alpha_2 & 0      & 1      & 0      & \ldots & 0      \\
    \alpha_3 & 0      & 0      & 1      & \ldots & 0      \\
    \vdots   & \vdots & \vdots & \vdots & \ddots & \vdots \\
    \alpha_n & 0      & 0      & 0      & \ldots & 0      \\
  \end{array}
  \right].
\end{displaymath}

Формула общего члена прогрессии имеет вид
\[ \vec{a}_k = \vec{a}_0 Q^k. \]


\section{Сумма геометрической прогрессии}

Найдём сумму первых $k$ элементов прогессии
\[ \vec{s}_k = \sum_{i=0}^{k-1}\vec{a}_i. \]

Повторим рассуждения Гауса
\[ \vec{s}_k(Q - E) = \vec{s}_k Q - \vec{s}_k = \vec{a}_k - \vec{a}_0, \]
следовательно если матрица $(Q - E)^{-1}$ обратима, то
\[ \vec{s}_k = (\vec{a}_k - \vec{a}_0) (Q - E)^{-1}. \]


\section{Об обратимости}

Укажем условие при котором матрица $Q - E$ обратима. Найдём определитель этой матрицы
\begin{displaymath}
  |Q - E| = \left|
  \begin{array}{cccccc}
    \alpha_1-1 & 1      & 0      & 0      & \ldots & 0      \\
    \alpha_2   & -1     & 1      & 0      & \ldots & 0      \\
    \alpha_3   & 0      & -1     & 1      & \ldots & 0      \\
    \vdots     & \vdots & \vdots & \vdots & \ddots & \vdots \\
    \alpha_n   & 0      & 0      & 0      & \ldots & -1     \\
  \end{array}
  \right| = \ldots
\end{displaymath}
распишем матрицу по первой строке
\begin{displaymath}
  \ldots = (\alpha_1 - 1) \left|
  \begin{array}{ccccc}
    -1     & 1      & 0      & \ldots & 0      \\
    0      & -1     & 1      & \ldots & 0      \\
    \vdots & \vdots & \vdots & \ddots & \vdots \\
    0      & 0      & 0      & \ldots & -1     \\
  \end{array}
  \right|
  -
  \left|
  \begin{array}{ccccc}
    \alpha_2   & 1      & 0      & \ldots & 0      \\
    \alpha_3   & -1     & 1      & \ldots & 0      \\
    \vdots     & \vdots & \vdots & \ddots & \vdots \\
    \alpha_n   & 0      & 0      & \ldots & -1     \\
  \end{array}
  \right| = \ldots
\end{displaymath}
определитель левой матрицы равен $(-1)^{n-1}$, правую матрицу вновь распишем по первой строке
\begin{displaymath}
  \ldots = (-1)^{n-1}(\alpha_1 - 1) 
  -
  \alpha_2\left|
  \begin{array}{cccc}
    -1     & 1      & \ldots & 0      \\
    \vdots & \vdots & \ddots & \vdots \\
    0      & 0      & \ldots & -1     \\
  \end{array}
  \right|
  +
  \left|
  \begin{array}{cccc}
    \alpha_3   & 1      & \ldots & 0      \\
    \vdots     & \vdots & \ddots & \vdots \\
    \alpha_n   & 0      & \ldots & -1     \\
  \end{array}
  \right| = \ldots,
\end{displaymath}
продолжая по аналогии, получим
\[
  \ldots = (-1)^{n-1}(\alpha_1 - 1) - (-1)^{n-2}\alpha_2 + (-1)^{n-3}\alpha_3 - \ldots - (-1)^{n}\alpha_n,
\]
или, что тоже самое,
\[
|Q - E| = (-1)^{n} (1 - \sum_{i=1}^{n}\alpha_i).
\]

Таким образом, мы получаем следующее условие на существование обратной матрицы: $\sum_{i=1}^{n}\alpha_i \not= 1$.


В заключении, осталось отметить лишь то, что сумма первых $k$ членов исходной рекуррентной последовательности равна первой компаненте $\vec{s}_k$.


\section{О сумме геометрической прогрессии над кольцами}

Укажем не сложную формулу для рассчёта суммы геометрической прогрессии:

\begin{displaymath}
  (\vec{a}_{k}, \vec{s}_{k}) = (\vec{a}_0, 0) \left[
  \begin{array}{cc}
    Q & 1 \\
    0 & 1 \\
  \end{array}
  \right] ^ k.
\end{displaymath}

\end{document}
